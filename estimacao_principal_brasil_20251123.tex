% Options for packages loaded elsewhere
\PassOptionsToPackage{unicode}{hyperref}
\PassOptionsToPackage{hyphens}{url}
\PassOptionsToPackage{dvipsnames,svgnames,x11names}{xcolor}
%
\documentclass[
  letterpaper,
  DIV=11,
  numbers=noendperiod]{scrartcl}

\usepackage{amsmath,amssymb}
\usepackage{iftex}
\ifPDFTeX
  \usepackage[T1]{fontenc}
  \usepackage[utf8]{inputenc}
  \usepackage{textcomp} % provide euro and other symbols
\else % if luatex or xetex
  \usepackage{unicode-math}
  \defaultfontfeatures{Scale=MatchLowercase}
  \defaultfontfeatures[\rmfamily]{Ligatures=TeX,Scale=1}
\fi
\usepackage{lmodern}
\ifPDFTeX\else  
    % xetex/luatex font selection
\fi
% Use upquote if available, for straight quotes in verbatim environments
\IfFileExists{upquote.sty}{\usepackage{upquote}}{}
\IfFileExists{microtype.sty}{% use microtype if available
  \usepackage[]{microtype}
  \UseMicrotypeSet[protrusion]{basicmath} % disable protrusion for tt fonts
}{}
\makeatletter
\@ifundefined{KOMAClassName}{% if non-KOMA class
  \IfFileExists{parskip.sty}{%
    \usepackage{parskip}
  }{% else
    \setlength{\parindent}{0pt}
    \setlength{\parskip}{6pt plus 2pt minus 1pt}}
}{% if KOMA class
  \KOMAoptions{parskip=half}}
\makeatother
\usepackage{xcolor}
\setlength{\emergencystretch}{3em} % prevent overfull lines
\setcounter{secnumdepth}{-\maxdimen} % remove section numbering
% Make \paragraph and \subparagraph free-standing
\makeatletter
\ifx\paragraph\undefined\else
  \let\oldparagraph\paragraph
  \renewcommand{\paragraph}{
    \@ifstar
      \xxxParagraphStar
      \xxxParagraphNoStar
  }
  \newcommand{\xxxParagraphStar}[1]{\oldparagraph*{#1}\mbox{}}
  \newcommand{\xxxParagraphNoStar}[1]{\oldparagraph{#1}\mbox{}}
\fi
\ifx\subparagraph\undefined\else
  \let\oldsubparagraph\subparagraph
  \renewcommand{\subparagraph}{
    \@ifstar
      \xxxSubParagraphStar
      \xxxSubParagraphNoStar
  }
  \newcommand{\xxxSubParagraphStar}[1]{\oldsubparagraph*{#1}\mbox{}}
  \newcommand{\xxxSubParagraphNoStar}[1]{\oldsubparagraph{#1}\mbox{}}
\fi
\makeatother

\usepackage{color}
\usepackage{fancyvrb}
\newcommand{\VerbBar}{|}
\newcommand{\VERB}{\Verb[commandchars=\\\{\}]}
\DefineVerbatimEnvironment{Highlighting}{Verbatim}{commandchars=\\\{\}}
% Add ',fontsize=\small' for more characters per line
\usepackage{framed}
\definecolor{shadecolor}{RGB}{241,243,245}
\newenvironment{Shaded}{\begin{snugshade}}{\end{snugshade}}
\newcommand{\AlertTok}[1]{\textcolor[rgb]{0.68,0.00,0.00}{#1}}
\newcommand{\AnnotationTok}[1]{\textcolor[rgb]{0.37,0.37,0.37}{#1}}
\newcommand{\AttributeTok}[1]{\textcolor[rgb]{0.40,0.45,0.13}{#1}}
\newcommand{\BaseNTok}[1]{\textcolor[rgb]{0.68,0.00,0.00}{#1}}
\newcommand{\BuiltInTok}[1]{\textcolor[rgb]{0.00,0.23,0.31}{#1}}
\newcommand{\CharTok}[1]{\textcolor[rgb]{0.13,0.47,0.30}{#1}}
\newcommand{\CommentTok}[1]{\textcolor[rgb]{0.37,0.37,0.37}{#1}}
\newcommand{\CommentVarTok}[1]{\textcolor[rgb]{0.37,0.37,0.37}{\textit{#1}}}
\newcommand{\ConstantTok}[1]{\textcolor[rgb]{0.56,0.35,0.01}{#1}}
\newcommand{\ControlFlowTok}[1]{\textcolor[rgb]{0.00,0.23,0.31}{\textbf{#1}}}
\newcommand{\DataTypeTok}[1]{\textcolor[rgb]{0.68,0.00,0.00}{#1}}
\newcommand{\DecValTok}[1]{\textcolor[rgb]{0.68,0.00,0.00}{#1}}
\newcommand{\DocumentationTok}[1]{\textcolor[rgb]{0.37,0.37,0.37}{\textit{#1}}}
\newcommand{\ErrorTok}[1]{\textcolor[rgb]{0.68,0.00,0.00}{#1}}
\newcommand{\ExtensionTok}[1]{\textcolor[rgb]{0.00,0.23,0.31}{#1}}
\newcommand{\FloatTok}[1]{\textcolor[rgb]{0.68,0.00,0.00}{#1}}
\newcommand{\FunctionTok}[1]{\textcolor[rgb]{0.28,0.35,0.67}{#1}}
\newcommand{\ImportTok}[1]{\textcolor[rgb]{0.00,0.46,0.62}{#1}}
\newcommand{\InformationTok}[1]{\textcolor[rgb]{0.37,0.37,0.37}{#1}}
\newcommand{\KeywordTok}[1]{\textcolor[rgb]{0.00,0.23,0.31}{\textbf{#1}}}
\newcommand{\NormalTok}[1]{\textcolor[rgb]{0.00,0.23,0.31}{#1}}
\newcommand{\OperatorTok}[1]{\textcolor[rgb]{0.37,0.37,0.37}{#1}}
\newcommand{\OtherTok}[1]{\textcolor[rgb]{0.00,0.23,0.31}{#1}}
\newcommand{\PreprocessorTok}[1]{\textcolor[rgb]{0.68,0.00,0.00}{#1}}
\newcommand{\RegionMarkerTok}[1]{\textcolor[rgb]{0.00,0.23,0.31}{#1}}
\newcommand{\SpecialCharTok}[1]{\textcolor[rgb]{0.37,0.37,0.37}{#1}}
\newcommand{\SpecialStringTok}[1]{\textcolor[rgb]{0.13,0.47,0.30}{#1}}
\newcommand{\StringTok}[1]{\textcolor[rgb]{0.13,0.47,0.30}{#1}}
\newcommand{\VariableTok}[1]{\textcolor[rgb]{0.07,0.07,0.07}{#1}}
\newcommand{\VerbatimStringTok}[1]{\textcolor[rgb]{0.13,0.47,0.30}{#1}}
\newcommand{\WarningTok}[1]{\textcolor[rgb]{0.37,0.37,0.37}{\textit{#1}}}

\providecommand{\tightlist}{%
  \setlength{\itemsep}{0pt}\setlength{\parskip}{0pt}}\usepackage{longtable,booktabs,array}
\usepackage{calc} % for calculating minipage widths
% Correct order of tables after \paragraph or \subparagraph
\usepackage{etoolbox}
\makeatletter
\patchcmd\longtable{\par}{\if@noskipsec\mbox{}\fi\par}{}{}
\makeatother
% Allow footnotes in longtable head/foot
\IfFileExists{footnotehyper.sty}{\usepackage{footnotehyper}}{\usepackage{footnote}}
\makesavenoteenv{longtable}
\usepackage{graphicx}
\makeatletter
\def\maxwidth{\ifdim\Gin@nat@width>\linewidth\linewidth\else\Gin@nat@width\fi}
\def\maxheight{\ifdim\Gin@nat@height>\textheight\textheight\else\Gin@nat@height\fi}
\makeatother
% Scale images if necessary, so that they will not overflow the page
% margins by default, and it is still possible to overwrite the defaults
% using explicit options in \includegraphics[width, height, ...]{}
\setkeys{Gin}{width=\maxwidth,height=\maxheight,keepaspectratio}
% Set default figure placement to htbp
\makeatletter
\def\fps@figure{htbp}
\makeatother

\usepackage{float}
\usepackage{tabularray}
\usepackage[normalem]{ulem}
\usepackage{graphicx}
\UseTblrLibrary{booktabs}
\UseTblrLibrary{rotating}
\UseTblrLibrary{siunitx}
\NewTableCommand{\tinytableDefineColor}[3]{\definecolor{#1}{#2}{#3}}
\newcommand{\tinytableTabularrayUnderline}[1]{\underline{#1}}
\newcommand{\tinytableTabularrayStrikeout}[1]{\sout{#1}}
\usepackage{fvextra}
\DefineVerbatimEnvironment{Highlighting}{Verbatim}{breaklines,commandchars=\\\{\}}
\DefineVerbatimEnvironment{OutputCode}{Verbatim}{breaklines,commandchars=\\\{\}}
\KOMAoption{captions}{tableheading}
\makeatletter
\@ifpackageloaded{caption}{}{\usepackage{caption}}
\AtBeginDocument{%
\ifdefined\contentsname
  \renewcommand*\contentsname{Table of contents}
\else
  \newcommand\contentsname{Table of contents}
\fi
\ifdefined\listfigurename
  \renewcommand*\listfigurename{List of Figures}
\else
  \newcommand\listfigurename{List of Figures}
\fi
\ifdefined\listtablename
  \renewcommand*\listtablename{List of Tables}
\else
  \newcommand\listtablename{List of Tables}
\fi
\ifdefined\figurename
  \renewcommand*\figurename{Figure}
\else
  \newcommand\figurename{Figure}
\fi
\ifdefined\tablename
  \renewcommand*\tablename{Table}
\else
  \newcommand\tablename{Table}
\fi
}
\@ifpackageloaded{float}{}{\usepackage{float}}
\floatstyle{ruled}
\@ifundefined{c@chapter}{\newfloat{codelisting}{h}{lop}}{\newfloat{codelisting}{h}{lop}[chapter]}
\floatname{codelisting}{Listing}
\newcommand*\listoflistings{\listof{codelisting}{List of Listings}}
\makeatother
\makeatletter
\makeatother
\makeatletter
\@ifpackageloaded{caption}{}{\usepackage{caption}}
\@ifpackageloaded{subcaption}{}{\usepackage{subcaption}}
\makeatother

\ifLuaTeX
  \usepackage{selnolig}  % disable illegal ligatures
\fi
\usepackage{bookmark}

\IfFileExists{xurl.sty}{\usepackage{xurl}}{} % add URL line breaks if available
\urlstyle{same} % disable monospaced font for URLs
\hypersetup{
  pdftitle={Trabalho de Avaliação de Políticas Sociais},
  pdfauthor={Ana Di Nur, Gabriel Figueiredo, Tiago Brancher},
  colorlinks=true,
  linkcolor={blue},
  filecolor={Maroon},
  citecolor={Blue},
  urlcolor={Blue},
  pdfcreator={LaTeX via pandoc}}


\title{\fontsize{16pt}{18pt}\selectfont Trabalho de Avaliação de
Políticas Sociais}
\author{Ana Di Nur, Gabriel Figueiredo, Tiago Brancher}
\date{}

\begin{document}
\maketitle


\begin{itemize}
\item
  Diferença em relação à versão de 14/11/2025: Vou juntar
  dt\_municipios\_eleicoes com dt\_emendas antes de cada exercício em
  vez de tentar juntar tudo no começo e ir alterando depois.
\item
  Diferença em relação à versão de 15/11/2025: Mudei os dados
  selecionados de características municipais (para serem mais recentes)
  e a fonte de dados de emendas parlamentares (para que o ano da emenda
  refletisse seu ano de empenho, não proposição)
\item
  Diferença em relação à versão de 21/11/2025: Filtrei por eleições
  acirradas só mais tarde para poder fazer os gráficos de
  (des)continuidades de características e valor empenhado por razão de
  votos do candidato a prefeito do PP; não analisei as emendas de
  bancada após constatar que elas são poucas
\item
  Diferença em relação à versão de 22/11/2025: Escolhi as janelas ótimas
  segundo o método de Calonico \emph{et al.} em vez de arbitrariamente e
  coloquei as análises quadráticas como teste de robustez e, como a
  janela ótima varia de acordo com controles e forma funcional, estimei
  um modelo linear sem controles, um modelo linear com controles e um
  modelo quadrático com controles separadamente (análogo ao que o
  Naércio fez no artigo dele)
\end{itemize}

\section{0) Preparativos}\label{preparativos}

\begin{Shaded}
\begin{Highlighting}[]
\CommentTok{\# Instalar pacotes}
\FunctionTok{library}\NormalTok{(readxl) }\CommentTok{\# para obtenção dos dados}
\FunctionTok{library}\NormalTok{(basedosdados) }\CommentTok{\# para obtenção dos dados}
\FunctionTok{library}\NormalTok{(dplyr) }\CommentTok{\# para transformação dos dados}
\FunctionTok{library}\NormalTok{(data.table) }\CommentTok{\# para transformação dos dados}
\FunctionTok{library}\NormalTok{(tibble) }\CommentTok{\# para transformação dos dados}
\FunctionTok{library}\NormalTok{(janitor) }\CommentTok{\# para transformação dos dados}
\FunctionTok{library}\NormalTok{(ggplot2) }\CommentTok{\# para visualização dos dados}
\FunctionTok{library}\NormalTok{(cowplot) }\CommentTok{\# para visualização dos dados (plotá{-}los lado a lado)}
\FunctionTok{library}\NormalTok{(scales) }\CommentTok{\# para visualização dos dados (fazer eixos bonito)}
\FunctionTok{library}\NormalTok{(showtext) }\CommentTok{\# para usar outras fontes}
\FunctionTok{library}\NormalTok{(estimatr) }\CommentTok{\# para modelagem}
\FunctionTok{library}\NormalTok{(rdrobust) }\CommentTok{\# para escolher a janela ótima}
\FunctionTok{library}\NormalTok{(modelsummary) }\CommentTok{\# para visualização de resultados dos modelos}

\CommentTok{\# Carregar fonte Computer Modern}
\FunctionTok{font\_add}\NormalTok{(}\StringTok{"CMU Serif"}\NormalTok{, }\StringTok{"Fontes/cmunrm.ttf"}\NormalTok{)}
\FunctionTok{showtext\_auto}\NormalTok{()}
\end{Highlighting}
\end{Shaded}

\section{1) Coleta e junção de
dados}\label{coleta-e-junuxe7uxe3o-de-dados}

\subsection{1.1) Coleta de dados}\label{coleta-de-dados}

\subsubsection{1.1.1) Características
municipais}\label{caracteruxedsticas-municipais}

\begin{itemize}
\item
  Origem dos dados: Acessar \href{https://cidades.ibge.gov.br/}{IBGE
  Cidades} \textgreater{} Selecionar UFs \textgreater{} Selecionar
  variáveis de interesse
\item
  Hipótese: Perfil municipal não mudou significativamente entre os anos
  de coleta dos dados (2021, 2022) e o ano da eleição (2020) e do
  empenho das emendas (2021)
\item
  Variáveis selecionadas:

  \begin{itemize}
  \item
    População no último censo (2022)
  \item
    Densidade\textbackslash ndemográfica (2022)
  \item
    PIB \emph{per capita} municipal (2021)
  \item
    Taxa de escolarização de 6 a 14 anos de idade (2022)
  \item
    Mortalidade infantil (2023)
  \end{itemize}
\end{itemize}

\begin{Shaded}
\begin{Highlighting}[]
\CommentTok{\# Carregar planilha montada a partir dos relatórios do IBGE Cidades}
\NormalTok{dt\_municipios }\OtherTok{\textless{}{-}} \FunctionTok{read\_excel}\NormalTok{(}\StringTok{"Bases de dados/Brasil/IBGE\_municipios.xlsx"}\NormalTok{)}
\FunctionTok{setDT}\NormalTok{(dt\_municipios)}

\CommentTok{\# Renomear colunas}
\NormalTok{nomes }\OtherTok{\textless{}{-}} \FunctionTok{c}\NormalTok{(}\StringTok{"id\_municipio\_nome"}\NormalTok{, }\StringTok{"sigla\_uf"}\NormalTok{, }\StringTok{"mortalidade\_infantil"}\NormalTok{, }\StringTok{"PIBpc"}\NormalTok{, }\StringTok{"taxa\_escolarizacao"}\NormalTok{, }\StringTok{"pop"}\NormalTok{, }\StringTok{"densidade\_demografica"}\NormalTok{)}
\FunctionTok{colnames}\NormalTok{(dt\_municipios) }\OtherTok{\textless{}{-}}\NormalTok{ nomes}
\FunctionTok{rm}\NormalTok{(nomes)}

\CommentTok{\# Transformar gentílico na sigla do UF}
\NormalTok{dt\_municipios[, sigla\_uf }\SpecialCharTok{:}\ErrorTok{=} \FunctionTok{fcase}\NormalTok{(}
\NormalTok{  sigla\_uf }\SpecialCharTok{==} \StringTok{"acriano"}\NormalTok{, }\StringTok{"AC"}\NormalTok{,}
\NormalTok{  sigla\_uf }\SpecialCharTok{==} \StringTok{"alagoano"}\NormalTok{, }\StringTok{"AL"}\NormalTok{,}
\NormalTok{  sigla\_uf }\SpecialCharTok{==} \StringTok{"amapaense"}\NormalTok{, }\StringTok{"AP"}\NormalTok{,}
\NormalTok{  sigla\_uf }\SpecialCharTok{==} \StringTok{"amazonense"}\NormalTok{, }\StringTok{"AM"}\NormalTok{,}
\NormalTok{  sigla\_uf }\SpecialCharTok{==} \StringTok{"baiano"}\NormalTok{, }\StringTok{"BA"}\NormalTok{,}
\NormalTok{  sigla\_uf }\SpecialCharTok{==} \StringTok{"cearense"}\NormalTok{, }\StringTok{"CE"}\NormalTok{,}
\NormalTok{  sigla\_uf }\SpecialCharTok{==} \StringTok{"brasiliense"}\NormalTok{, }\StringTok{"DF"}\NormalTok{,}
\NormalTok{  sigla\_uf }\SpecialCharTok{==} \StringTok{"capixaba ou espírito{-}santense"}\NormalTok{, }\StringTok{"ES"}\NormalTok{,}
\NormalTok{  sigla\_uf }\SpecialCharTok{==} \StringTok{"goiano"}\NormalTok{, }\StringTok{"GO"}\NormalTok{,}
\NormalTok{  sigla\_uf }\SpecialCharTok{==} \StringTok{"maranhense"}\NormalTok{, }\StringTok{"MA"}\NormalTok{,}
\NormalTok{  sigla\_uf }\SpecialCharTok{==} \StringTok{"mato{-}grossense"}\NormalTok{, }\StringTok{"MT"}\NormalTok{,}
\NormalTok{  sigla\_uf }\SpecialCharTok{==} \StringTok{"sul{-}mato{-}grossense ou mato{-}grossense{-}do{-}sul"}\NormalTok{, }\StringTok{"MS"}\NormalTok{,}
\NormalTok{  sigla\_uf }\SpecialCharTok{==} \StringTok{"mineiro"}\NormalTok{, }\StringTok{"MG"}\NormalTok{,}
\NormalTok{  sigla\_uf }\SpecialCharTok{==} \StringTok{"paranaense"}\NormalTok{, }\StringTok{"PR"}\NormalTok{,}
\NormalTok{  sigla\_uf }\SpecialCharTok{==} \StringTok{"paraibano"}\NormalTok{, }\StringTok{"PB"}\NormalTok{,}
\NormalTok{  sigla\_uf }\SpecialCharTok{==} \StringTok{"paraense"}\NormalTok{, }\StringTok{"PA"}\NormalTok{,}
\NormalTok{  sigla\_uf }\SpecialCharTok{==} \StringTok{"pernambucano"}\NormalTok{, }\StringTok{"PE"}\NormalTok{,}
\NormalTok{  sigla\_uf }\SpecialCharTok{==} \StringTok{"piauiense"}\NormalTok{, }\StringTok{"PI"}\NormalTok{,}
\NormalTok{  sigla\_uf }\SpecialCharTok{==} \StringTok{"potiguar, norte{-}rio{-}grandense, rio{-}grandense{-}do{-}norte"}\NormalTok{, }\StringTok{"RN"}\NormalTok{,}
\NormalTok{  sigla\_uf }\SpecialCharTok{==} \StringTok{"gaúcho ou sul{-}rio{-}grandense"}\NormalTok{, }\StringTok{"RS"}\NormalTok{,}
\NormalTok{  sigla\_uf }\SpecialCharTok{==} \StringTok{"fluminense"}\NormalTok{, }\StringTok{"RJ"}\NormalTok{,}
\NormalTok{  sigla\_uf }\SpecialCharTok{==} \StringTok{"rondoniense ou rondoniano"}\NormalTok{, }\StringTok{"RO"}\NormalTok{,}
\NormalTok{  sigla\_uf }\SpecialCharTok{==} \StringTok{"roraimense"}\NormalTok{, }\StringTok{"RR"}\NormalTok{,}
\NormalTok{  sigla\_uf }\SpecialCharTok{==} \StringTok{"catarinense ou barriga{-}verde"}\NormalTok{, }\StringTok{"SC"}\NormalTok{,}
\NormalTok{  sigla\_uf }\SpecialCharTok{==} \StringTok{"sergipano ou sergipense"}\NormalTok{, }\StringTok{"SE"}\NormalTok{,}
\NormalTok{  sigla\_uf }\SpecialCharTok{==} \StringTok{"paulista"}\NormalTok{, }\StringTok{"SP"}\NormalTok{,}
\NormalTok{  sigla\_uf }\SpecialCharTok{==} \StringTok{"tocantinense"}\NormalTok{, }\StringTok{"TO"}
\NormalTok{)]}

\CommentTok{\# Criar coluna de região}
\NormalTok{dt\_municipios[, regiao }\SpecialCharTok{:}\ErrorTok{=} \FunctionTok{fcase}\NormalTok{(}
\NormalTok{    sigla\_uf }\SpecialCharTok{\%in\%} \FunctionTok{c}\NormalTok{(}\StringTok{"AC"}\NormalTok{, }\StringTok{"AM"}\NormalTok{, }\StringTok{"AP"}\NormalTok{, }\StringTok{"PA"}\NormalTok{, }\StringTok{"RO"}\NormalTok{, }\StringTok{"RR"}\NormalTok{, }\StringTok{"TO"}\NormalTok{), }\StringTok{"Norte"}\NormalTok{,}
\NormalTok{    sigla\_uf }\SpecialCharTok{\%in\%} \FunctionTok{c}\NormalTok{(}\StringTok{"AL"}\NormalTok{, }\StringTok{"BA"}\NormalTok{, }\StringTok{"CE"}\NormalTok{, }\StringTok{"MA"}\NormalTok{, }\StringTok{"PE"}\NormalTok{, }\StringTok{"PB"}\NormalTok{, }\StringTok{"PI"}\NormalTok{, }\StringTok{"RN"}\NormalTok{, }\StringTok{"SE"}\NormalTok{), }\StringTok{"Nordeste"}\NormalTok{,}
\NormalTok{    sigla\_uf }\SpecialCharTok{\%in\%} \FunctionTok{c}\NormalTok{(}\StringTok{"DF"}\NormalTok{, }\StringTok{"GO"}\NormalTok{, }\StringTok{"MS"}\NormalTok{, }\StringTok{"MT"}\NormalTok{), }\StringTok{"Centro{-}Oeste"}\NormalTok{,}
\NormalTok{    sigla\_uf }\SpecialCharTok{\%in\%} \FunctionTok{c}\NormalTok{(}\StringTok{"ES"}\NormalTok{, }\StringTok{"MG"}\NormalTok{, }\StringTok{"RJ"}\NormalTok{, }\StringTok{"SP"}\NormalTok{), }\StringTok{"Sudeste"}\NormalTok{,}
\NormalTok{    sigla\_uf }\SpecialCharTok{\%in\%} \FunctionTok{c}\NormalTok{(}\StringTok{"PR"}\NormalTok{, }\StringTok{"RS"}\NormalTok{, }\StringTok{"SC"}\NormalTok{), }\StringTok{"Sul"}\NormalTok{)]}

\CommentTok{\# Transformar colunas em variáveis numéricas}
\DocumentationTok{\#\# Inteiros}
\NormalTok{dt\_municipios[, pop }\SpecialCharTok{:}\ErrorTok{=} \FunctionTok{gsub}\NormalTok{(}\StringTok{"[\^{}0{-}9.{-}]"}\NormalTok{, }\StringTok{""}\NormalTok{, pop) }\SpecialCharTok{\%\textgreater{}\%} \FunctionTok{as.numeric}\NormalTok{()]}
\DocumentationTok{\#\# Decimais}
\NormalTok{dt\_municipios[,}
\NormalTok{  densidade\_demografica }\SpecialCharTok{:}\ErrorTok{=} \FunctionTok{as.numeric}\NormalTok{(}
    \FunctionTok{gsub}\NormalTok{(}\StringTok{","}\NormalTok{, }\StringTok{"."}\NormalTok{, }\FunctionTok{gsub}\NormalTok{(}\StringTok{"[\^{}0{-}9,]"}\NormalTok{, }\StringTok{""}\NormalTok{, densidade\_demografica))}
\NormalTok{  )}
\NormalTok{]}
\NormalTok{dt\_municipios[,}
\NormalTok{  mortalidade\_infantil }\SpecialCharTok{:}\ErrorTok{=} \FunctionTok{as.numeric}\NormalTok{(}
    \FunctionTok{gsub}\NormalTok{(}\StringTok{","}\NormalTok{, }\StringTok{"."}\NormalTok{, }\FunctionTok{gsub}\NormalTok{(}\StringTok{"[\^{}0{-}9,]"}\NormalTok{, }\StringTok{""}\NormalTok{, mortalidade\_infantil))}
\NormalTok{  )}
\NormalTok{]}
\NormalTok{dt\_municipios[,}
\NormalTok{  PIBpc }\SpecialCharTok{:}\ErrorTok{=} \FunctionTok{as.numeric}\NormalTok{(}
    \FunctionTok{gsub}\NormalTok{(}\StringTok{","}\NormalTok{, }\StringTok{"."}\NormalTok{, }\FunctionTok{gsub}\NormalTok{(}\StringTok{"[\^{}0{-}9,]"}\NormalTok{, }\StringTok{""}\NormalTok{, PIBpc))}
\NormalTok{  )}
\NormalTok{]}
\NormalTok{dt\_municipios[,}
\NormalTok{  taxa\_escolarizacao }\SpecialCharTok{:}\ErrorTok{=} \FunctionTok{as.numeric}\NormalTok{(}
    \FunctionTok{gsub}\NormalTok{(}\StringTok{","}\NormalTok{, }\StringTok{"."}\NormalTok{, }\FunctionTok{gsub}\NormalTok{(}\StringTok{"[\^{}0{-}9,]"}\NormalTok{, }\StringTok{""}\NormalTok{, taxa\_escolarizacao))}
\NormalTok{  )}\SpecialCharTok{/}\DecValTok{100}
\NormalTok{]}

\CommentTok{\# Dar uma olhada na base}
\FunctionTok{str}\NormalTok{(dt\_municipios)}
\end{Highlighting}
\end{Shaded}

\begin{verbatim}
Classes 'data.table' and 'data.frame':  5571 obs. of  8 variables:
 $ id_municipio_nome    : chr  "Acrelândia" "Assis Brasil" "Brasiléia" "Bujari" ...
 $ sigla_uf             : chr  "AC" "AC" "AC" "AC" ...
 $ mortalidade_infantil : num  NA 7.78 5.78 19.53 11.11 ...
 $ PIBpc                : num  25363 17508 25279 28455 41723 ...
 $ taxa_escolarizacao   : num  0.961 0.957 0.959 0.957 0.96 ...
 $ pop                  : num  14021 8100 26000 12917 10392 ...
 $ densidade_demografica: num  7.74 1.63 6.62 4.26 6.09 ...
 $ regiao               : chr  "Norte" "Norte" "Norte" "Norte" ...
 - attr(*, ".internal.selfref")=<externalptr> 
\end{verbatim}

\subsubsection{1.1.2) Eleições}\label{eleiuxe7uxf5es}

\begin{itemize}
\tightlist
\item
  Origem dos dados: Acessar
  \href{https://basedosdados.org/dataset/eef764df-bde8-4905-b115-6fc23b6ba9d6?table=391047eb-b3ef-4141-a4d1-725b29018f25}{Base
  dos Dados \textgreater{} Eleições Brasileiras} \textgreater{}
  Resultados por Candidato e Município
\end{itemize}

\begin{Shaded}
\begin{Highlighting}[]
\CommentTok{\# \# Fazer query}
\CommentTok{\# query \textless{}{-} "SELECT}
\CommentTok{\#     dados.ano AS ano,}
\CommentTok{\#     dados.turno AS turno,}
\CommentTok{\#     dados.sigla\_uf AS sigla\_uf,}
\CommentTok{\#     diretorio\_sigla\_uf.nome AS sigla\_uf\_nome,}
\CommentTok{\#     dados.id\_municipio AS id\_municipio,}
\CommentTok{\#     diretorio\_id\_municipio.nome AS id\_municipio\_nome,}
\CommentTok{\#     dados.cargo AS cargo,}
\CommentTok{\#     dados.numero\_partido AS numero\_partido,}
\CommentTok{\#     dados.sigla\_partido AS sigla\_partido,}
\CommentTok{\#     dados.resultado AS resultado,}
\CommentTok{\#     dados.votos AS votos}
\CommentTok{\# FROM \textasciigrave{}basedosdados.br\_tse\_eleicoes.resultados\_candidato\_municipio\textasciigrave{} AS dados}
\CommentTok{\# LEFT JOIN (}
\CommentTok{\#     SELECT DISTINCT sigla, nome  }
\CommentTok{\#     FROM \textasciigrave{}basedosdados.br\_bd\_diretorios\_brasil.uf\textasciigrave{}}
\CommentTok{\# ) AS diretorio\_sigla\_uf}
\CommentTok{\#     ON dados.sigla\_uf = diretorio\_sigla\_uf.sigla}
\CommentTok{\# LEFT JOIN (}
\CommentTok{\#     SELECT DISTINCT id\_municipio, nome  }
\CommentTok{\#     FROM \textasciigrave{}basedosdados.br\_bd\_diretorios\_brasil.municipio\textasciigrave{}}
\CommentTok{\# ) AS diretorio\_id\_municipio}
\CommentTok{\#     ON dados.id\_municipio = diretorio\_id\_municipio.id\_municipio}
\CommentTok{\# WHERE }
\CommentTok{\#     dados.ano IN (2020)}
\CommentTok{\#     AND dados.cargo = \textquotesingle{}prefeito\textquotesingle{}}
\CommentTok{\# "}
\CommentTok{\# dt\_eleicoes \textless{}{-} read\_sql(query, billing\_project\_id = "pub{-}450900")}
\CommentTok{\# setDT(dt\_eleicoes)}
\CommentTok{\# rm(query)}
\CommentTok{\# }
\CommentTok{\# \# Salvar base de dados resultante}
\CommentTok{\# saveRDS(dt\_eleicoes, "Bases de dados/Brasil/TSE\_eleicoes.rds")}


\CommentTok{\# Carregar base de dados de eleições}
\NormalTok{dt\_eleicoes }\OtherTok{\textless{}{-}} \FunctionTok{readRDS}\NormalTok{(}\StringTok{"Bases de dados/Brasil/TSE\_eleicoes.rds"}\NormalTok{)}
\FunctionTok{setDT}\NormalTok{(dt\_eleicoes)}

\CommentTok{\# Retirar municípios que tiveram eleições suplementares (identificados pela presença de mais de um candidato eleito na base) e Cachoeira dos Índios, que parece ter passado por algo parecido}
\NormalTok{municipios\_com\_mais\_de\_um\_eleito }\OtherTok{\textless{}{-}}\NormalTok{ dt\_eleicoes[resultado }\SpecialCharTok{==} \StringTok{"eleito"}\NormalTok{, .N, by }\OtherTok{=}\NormalTok{ id\_municipio][N}\SpecialCharTok{\textgreater{}}\DecValTok{1}\NormalTok{, }\FunctionTok{unique}\NormalTok{(id\_municipio)]}
\NormalTok{dt\_eleicoes }\OtherTok{\textless{}{-}}\NormalTok{ dt\_eleicoes[}\SpecialCharTok{!}\NormalTok{(id\_municipio }\SpecialCharTok{\%in\%}\NormalTok{ municipios\_com\_mais\_de\_um\_eleito) }\SpecialCharTok{\&}\NormalTok{ id\_municipio }\SpecialCharTok{!=} \DecValTok{2503308}\NormalTok{]}
\FunctionTok{rm}\NormalTok{(municipios\_com\_mais\_de\_um\_eleito)}

\CommentTok{\# Retirar primeiro turno dos municípios que tiveram segundo turno}
\NormalTok{municipios\_com\_segundo\_turno }\OtherTok{\textless{}{-}}\NormalTok{ dt\_eleicoes[turno }\SpecialCharTok{==} \DecValTok{2}\NormalTok{, }\FunctionTok{unique}\NormalTok{(id\_municipio)]}
\NormalTok{dt\_eleicoes }\OtherTok{\textless{}{-}}\NormalTok{ dt\_eleicoes[}\SpecialCharTok{!}\NormalTok{(turno }\SpecialCharTok{==} \DecValTok{1} \SpecialCharTok{\&}
\NormalTok{                               id\_municipio }\SpecialCharTok{\%in\%}\NormalTok{ municipios\_com\_segundo\_turno)]}
\FunctionTok{rm}\NormalTok{(municipios\_com\_segundo\_turno)}

\CommentTok{\# Selecionar os dois candidatos mais votados}
\FunctionTok{setorder}\NormalTok{(dt\_eleicoes, }\SpecialCharTok{{-}}\NormalTok{votos)}
\NormalTok{dt\_eleicoes }\OtherTok{\textless{}{-}}\NormalTok{ dt\_eleicoes[, }\FunctionTok{head}\NormalTok{(.SD, }\DecValTok{2}\NormalTok{), by }\OtherTok{=}\NormalTok{ id\_municipio]}

\CommentTok{\# Definir razão de votos dos top 2 (contando somente os votos desses 2)}
\NormalTok{dt\_eleicoes[, votos\_total }\SpecialCharTok{:}\ErrorTok{=} \FunctionTok{sum}\NormalTok{(votos), by }\OtherTok{=}\NormalTok{ id\_municipio]}
\NormalTok{dt\_eleicoes[, votos\_razao }\SpecialCharTok{:}\ErrorTok{=}\NormalTok{ votos}\SpecialCharTok{/}\NormalTok{votos\_total]}

\CommentTok{\# Retirar municípios em que nenhum dos dois primeiros colocados são do PP}
\NormalTok{municipios\_com\_candidato\_pp }\OtherTok{\textless{}{-}}\NormalTok{ dt\_eleicoes[sigla\_partido }\SpecialCharTok{==} \StringTok{"PP"}\NormalTok{, }\FunctionTok{unique}\NormalTok{(id\_municipio)]}
\NormalTok{dt\_eleicoes }\OtherTok{\textless{}{-}}\NormalTok{ dt\_eleicoes[id\_municipio }\SpecialCharTok{\%in\%}\NormalTok{ municipios\_com\_candidato\_pp]}
\FunctionTok{rm}\NormalTok{(municipios\_com\_candidato\_pp)}

\CommentTok{\# Criar dummy que indica se o candidato do PP ganhou}
\NormalTok{municipios\_com\_vitoria\_pp }\OtherTok{\textless{}{-}}\NormalTok{ dt\_eleicoes[sigla\_partido }\SpecialCharTok{==} \StringTok{"PP"} \SpecialCharTok{\&}\NormalTok{ resultado }\SpecialCharTok{==} \StringTok{"eleito"}\NormalTok{,}
                                         \FunctionTok{unique}\NormalTok{(id\_municipio)]}
\NormalTok{dt\_eleicoes[, vitoria\_pp }\SpecialCharTok{:}\ErrorTok{=} \FunctionTok{ifelse}\NormalTok{(id\_municipio }\SpecialCharTok{\%in\%}\NormalTok{ municipios\_com\_vitoria\_pp, }\DecValTok{1}\NormalTok{, }\DecValTok{0}\NormalTok{)]}
\FunctionTok{rm}\NormalTok{(municipios\_com\_vitoria\_pp)}

\CommentTok{\# Selecionar só colunas e linhas relevantes}
\NormalTok{dt\_eleicoes }\OtherTok{\textless{}{-}}\NormalTok{ dt\_eleicoes[}
\NormalTok{  sigla\_partido }\SpecialCharTok{==} \StringTok{"PP"}\NormalTok{,}
\NormalTok{  .(}
\NormalTok{    id\_municipio,}
\NormalTok{    id\_municipio\_nome,}
\NormalTok{    sigla\_uf,}
\NormalTok{    vitoria\_pp,}
\NormalTok{    votos\_razao}
\NormalTok{  )}
\NormalTok{]}
\FunctionTok{setnames}\NormalTok{(dt\_eleicoes, }\StringTok{"votos\_razao"}\NormalTok{, }\StringTok{"votos\_razao\_pp"}\NormalTok{)}

\CommentTok{\# Criar coluna com razão de votos centralizada}
\NormalTok{dt\_eleicoes[, votos\_razao\_pp\_centr }\SpecialCharTok{:}\ErrorTok{=}\NormalTok{ votos\_razao\_pp }\SpecialCharTok{{-}} \FloatTok{0.5}\NormalTok{]}

\CommentTok{\# Criar coluna com razão de votos centralizada ao quadrado}
\NormalTok{dt\_eleicoes[, votos\_razao\_pp\_centr\_sq }\SpecialCharTok{:}\ErrorTok{=}\NormalTok{ votos\_razao\_pp\_centr}\SpecialCharTok{\^{}}\DecValTok{2}\NormalTok{]}

\CommentTok{\# Dar uma olhada na base de dados resultante}
\FunctionTok{str}\NormalTok{(dt\_eleicoes)}
\end{Highlighting}
\end{Shaded}

\begin{verbatim}
Classes 'data.table' and 'data.frame':  1196 obs. of  7 variables:
 $ id_municipio           : chr  "3170206" "3303500" "2507507" "3143302" ...
 $ id_municipio_nome      : chr  "Uberlândia" "Nova Iguaçu" "João Pessoa" "Montes Claros" ...
 $ sigla_uf               : chr  "MG" "RJ" "PB" "MG" ...
 $ vitoria_pp             : num  1 1 1 0 1 0 0 1 1 1 ...
 $ votos_razao_pp         : num  0.8523 0.8175 0.5316 0.0788 0.9127 ...
 $ votos_razao_pp_centr   : num  0.3523 0.3175 0.0316 -0.4212 0.4127 ...
 $ votos_razao_pp_centr_sq: num  0.124 0.101 0.001 0.177 0.17 ...
 - attr(*, ".internal.selfref")=<externalptr> 
\end{verbatim}

Note que:

\begin{itemize}
\item
  5567 municípios brasileiros tiveram eleições para prefeito em 2020;
\item
  1204 (22\% dos 5567) tiveram um candidato do PP entre os dois
  candidatos mais votados
\end{itemize}

\subsubsection{1.1.3) Emendas
parlamentares}\label{emendas-parlamentares}

\begin{itemize}
\item
  A base principal virá do Siga Brasil. No entanto, essa base não contém
  o código IBGE do município favorecido pela emenda. Por isso, vamos
  começar carregando dados da Base dos Dados (originados nos dados da
  CGU) que associam o identificador da emenda parlamentar ao município
  de destinação.
\item
  Origem dos dados:
  \href{https://basedosdados.org/dataset/257e000c-1685-418a-88d9-4908ccef2840?table=f116068d-b65d-4d04-9bcb-368e70062c4b}{Base
  dos Dados \textgreater{} Emendas Parlamentares}
\item
  Variáveis selecionadas:

  \begin{itemize}
  \tightlist
  \item
    ano\_emenda
  \end{itemize}

  *Filtro: ``2021''. Apesar de essa variável representar o ano em que a
  emenda foi proposta, não o ano do empenho, o Siga Brasil mostra que
  todas as 1183 emendas destinadas a municípios com empenho em 2021
  foram propostas em 2021.

  \begin{itemize}
  \tightlist
  \item
    id\_emenda
  \item
    numero\_emenda
  \item
    id\_municipio\_gasto
  \end{itemize}
\end{itemize}

\begin{Shaded}
\begin{Highlighting}[]
\CommentTok{\# \# Fazer query}
\CommentTok{\# query \textless{}{-} "}
\CommentTok{\# SELECT}
\CommentTok{\#     dados.id\_emenda as id\_emenda,}
\CommentTok{\#     dados.ano\_emenda as ano\_emenda,}
\CommentTok{\#     dados.id\_autor\_emenda as id\_autor\_emenda,}
\CommentTok{\#     dados.numero\_emenda as numero\_emenda,}
\CommentTok{\#     dados.id\_municipio\_gasto as id\_municipio}
\CommentTok{\# FROM \textasciigrave{}basedosdados.br\_cgu\_emendas\_parlamentares.microdados\textasciigrave{} AS dados}
\CommentTok{\# WHERE }
\CommentTok{\#     dados.ano\_emenda IN (2021)}
\CommentTok{\# "}
\CommentTok{\# de\_para\_emendas\_municipios \textless{}{-} read\_sql(query, billing\_project\_id = "pub{-}450900")}
\CommentTok{\# setDT(de\_para\_emendas\_municipios)}
\CommentTok{\# rm(query)}
\CommentTok{\# }
\CommentTok{\# \# Criar emenda (número/ano) igual à base de emendas do Siga Brasil}
\CommentTok{\# de\_para\_emendas\_municipios[, emenda\_numero\_ano := paste0(id\_autor\_emenda,}
\CommentTok{\#                                                          numero\_emenda,}
\CommentTok{\#                                                          "{-}",}
\CommentTok{\#                                                          ano\_emenda)]}
\CommentTok{\# }
\CommentTok{\# \# Selecionar somente as colunas de de{-}para}
\CommentTok{\# de\_para\_emendas\_municipios \textless{}{-} de\_para\_emendas\_municipios[, .(emenda\_numero\_ano, id\_municipio)]}
\CommentTok{\# }
\CommentTok{\# \# Salvar base de dados resultante}
\CommentTok{\# saveRDS(de\_para\_emendas\_municipios, "Bases de dados/Brasil/BD\_id\_emenda\_municipio.rds")}

\CommentTok{\# Carregar base de dados que relaciona emendas e municípios (por código do IBGE)}
\NormalTok{de\_para\_emendas\_municipios }\OtherTok{\textless{}{-}} \FunctionTok{readRDS}\NormalTok{(}\StringTok{"Bases de dados/Brasil/BD\_id\_emenda\_municipio.rds"}\NormalTok{)}
\end{Highlighting}
\end{Shaded}

\begin{itemize}
\item
  Origem dos dados:
  \href{https://www9qs.senado.leg.br/extensions/Siga_Brasil_Emendas/Siga_Brasil_Emendas.html}{Senado
  Federal \textgreater{} Portal do Orçamento \textgreater{} Siga Brasil
  \textgreater{} Painel Emendas} \textgreater{} Gráficos customizados

  \begin{itemize}
  \tightlist
  \item
    Variáveis selecionadas:
  \item
    Autor (Tipo)
  \item
    Emenda (Número-Ano)
  \item
    Empenho (Ano)
  \end{itemize}

  *Filtro: ``2021''

  \begin{itemize}
  \tightlist
  \item
    Função (Desc)
  \item
    Funcional Localidade (Desc)
  \item
    Funcional Localidade (Tipo)
  \end{itemize}

  *Filtro: ``MUNICÍPIO''

  \begin{itemize}
  \tightlist
  \item
    GND (Desc)
  \item
    Empenhado (IPCA)
  \end{itemize}
\end{itemize}

\begin{Shaded}
\begin{Highlighting}[]
\CommentTok{\# Carregar base de dados de emendas parlamentares}
\NormalTok{dt\_emendas }\OtherTok{\textless{}{-}} \FunctionTok{read\_excel}\NormalTok{(}\StringTok{"Bases de dados/Brasil/SigaBrasil\_emendas.xlsx"}\NormalTok{)}
\FunctionTok{setDT}\NormalTok{(dt\_emendas)}

\CommentTok{\# Renomear colunas}
\FunctionTok{colnames}\NormalTok{(dt\_emendas) }\OtherTok{\textless{}{-}} \FunctionTok{make\_clean\_names}\NormalTok{(}\FunctionTok{colnames}\NormalTok{(dt\_emendas))}

\CommentTok{\# Juntar com informação de códigos IBGE}
\NormalTok{dt\_emendas }\OtherTok{\textless{}{-}} \FunctionTok{merge}\NormalTok{(dt\_emendas, de\_para\_emendas\_municipios, }\AttributeTok{all.x =}\NormalTok{ T)}
\FunctionTok{rm}\NormalTok{(de\_para\_emendas\_municipios)}

\CommentTok{\# Substituir id\_municipio manualmente para as 4 emendas sem essa informação na base}
\NormalTok{dt\_emendas[emenda\_numero\_ano }\SpecialCharTok{==} \StringTok{"14510004{-}2021"}\NormalTok{ , id\_municipio }\SpecialCharTok{:}\ErrorTok{=} \StringTok{"5002704"}\NormalTok{]}
\NormalTok{dt\_emendas[emenda\_numero\_ano }\SpecialCharTok{==} \StringTok{"37390008{-}2021"}\NormalTok{, id\_municipio }\SpecialCharTok{:}\ErrorTok{=} \StringTok{"2504009"}\NormalTok{]}
\NormalTok{dt\_emendas[emenda\_numero\_ano }\SpecialCharTok{==} \StringTok{"40680005{-}2021"}\NormalTok{, id\_municipio }\SpecialCharTok{:}\ErrorTok{=} \StringTok{"1301605"}\NormalTok{]}

\CommentTok{\# Garantir que não há NAs}
\FunctionTok{colSums}\NormalTok{(}\FunctionTok{is.na}\NormalTok{(dt\_emendas))}
\end{Highlighting}
\end{Shaded}

\begin{verbatim}
        emenda_numero_ano                autor_tipo       ano_emissao_empenho 
                        0                         0                         0 
                   funcao      funcional_localidade funcional_localidade_tipo 
                        0                         0                         0 
                 gnd_desc            empenhado_ipca              id_municipio 
                        0                         0                         0 
\end{verbatim}

\begin{Shaded}
\begin{Highlighting}[]
\CommentTok{\# Renomear coluna de tipo}
\FunctionTok{setnames}\NormalTok{(dt\_emendas, }\StringTok{"autor\_tipo"}\NormalTok{, }\StringTok{"tipo\_emenda"}\NormalTok{)}

\CommentTok{\# Resumir dados por município, tipo de emenda, função e GND e armazenar dt para eventuais análises de heterogeneidade}
\NormalTok{dt\_emendas\_com\_funcao\_gnd }\OtherTok{\textless{}{-}}\NormalTok{ dt\_emendas[,}
\NormalTok{  .(}\AttributeTok{valor\_empenhado\_ipca =} \FunctionTok{sum}\NormalTok{(empenhado\_ipca)),}
\NormalTok{  by }\OtherTok{=}\NormalTok{ .(id\_municipio, tipo\_emenda, funcao, gnd\_desc)}
\NormalTok{]}

\CommentTok{\# Resumir dados por município e tipo de emenda}
\NormalTok{dt\_emendas }\OtherTok{\textless{}{-}}\NormalTok{ dt\_emendas[,}
\NormalTok{  .(}\AttributeTok{valor\_empenhado\_ipca =} \FunctionTok{sum}\NormalTok{(empenhado\_ipca)),}
\NormalTok{  by }\OtherTok{=}\NormalTok{ .(id\_municipio, tipo\_emenda)}
\NormalTok{]}

\CommentTok{\# Ver quais tipos de emendas há na base}
\NormalTok{dt\_emendas[, }\FunctionTok{unique}\NormalTok{(tipo\_emenda)]}
\end{Highlighting}
\end{Shaded}

\begin{verbatim}
[1] "INDIVIDUAL"              "BANCADA ESTADUAL (RP 7)"
[3] "RELATOR (RP 9)"         
\end{verbatim}

\begin{Shaded}
\begin{Highlighting}[]
\CommentTok{\# Simplificar nome do tipo de emenda}
\NormalTok{dt\_emendas[, tipo\_emenda }\SpecialCharTok{:}\ErrorTok{=} \FunctionTok{fcase}\NormalTok{(}
\NormalTok{  tipo\_emenda }\SpecialCharTok{==} \StringTok{"INDIVIDUAL"}\NormalTok{, }\StringTok{"individual"}\NormalTok{,}
\NormalTok{  tipo\_emenda }\SpecialCharTok{==} \StringTok{"BANCADA ESTADUAL (RP 7)"}\NormalTok{, }\StringTok{"bancada"}\NormalTok{,}
\NormalTok{  tipo\_emenda }\SpecialCharTok{==} \StringTok{"RELATOR (RP 9)"}\NormalTok{, }\StringTok{"relator"}
\NormalTok{)]}

\CommentTok{\# Dar uma olhada na base}
\FunctionTok{head}\NormalTok{(dt\_emendas)}
\end{Highlighting}
\end{Shaded}

\begin{verbatim}
   id_municipio tipo_emenda valor_empenhado_ipca
         <char>      <char>                <num>
1:      2607109  individual             122383.8
2:      2612604  individual             131423.8
3:      2607000  individual             252882.8
4:      2603009  individual             316103.5
5:      2210201  individual             255310.4
6:      2206670  individual             379324.2
\end{verbatim}

\begin{Shaded}
\begin{Highlighting}[]
\CommentTok{\# Resumir valor empenhado de emendas por tipo}
\NormalTok{dt\_emendas[, .(}\AttributeTok{valor\_empenhado\_ipca =} \FunctionTok{sum}\NormalTok{(valor\_empenhado\_ipca)), by }\OtherTok{=}\NormalTok{ tipo\_emenda]}
\end{Highlighting}
\end{Shaded}

\begin{verbatim}
   tipo_emenda valor_empenhado_ipca
        <char>                <num>
1:  individual            943753489
2:     bancada           1482525667
3:     relator             15760159
\end{verbatim}

\begin{Shaded}
\begin{Highlighting}[]
\CommentTok{\# Resumir qtde de municípios que receberam cada tipo de emenda}
\FunctionTok{writeLines}\NormalTok{(}\FunctionTok{paste0}\NormalTok{(}
  \StringTok{"Quantidade de municípios a receberem qualquer tipo de emenda: "}\NormalTok{, }\FunctionTok{length}\NormalTok{(dt\_emendas[, }\FunctionTok{unique}\NormalTok{(id\_municipio)]),}
  \StringTok{"}\SpecialCharTok{\textbackslash{}n}\StringTok{Quantidade de municípios a receberem emendas individuais: "}\NormalTok{, }\FunctionTok{length}\NormalTok{(dt\_emendas[tipo\_emenda }\SpecialCharTok{==} \StringTok{"individual"}\NormalTok{, }\FunctionTok{unique}\NormalTok{(id\_municipio)]),}
  \StringTok{"}\SpecialCharTok{\textbackslash{}n}\StringTok{Quantidade de municípios a receberem emendas de bancada: "}\NormalTok{, }\FunctionTok{length}\NormalTok{(dt\_emendas[tipo\_emenda }\SpecialCharTok{==} \StringTok{"bancada"}\NormalTok{, }\FunctionTok{unique}\NormalTok{(id\_municipio)]),}
  \StringTok{"}\SpecialCharTok{\textbackslash{}n}\StringTok{Quantidade de municípios a receberem emendas de relator: "}\NormalTok{, }\FunctionTok{length}\NormalTok{(dt\_emendas[tipo\_emenda }\SpecialCharTok{==} \StringTok{"relator"}\NormalTok{, }\FunctionTok{unique}\NormalTok{(id\_municipio)])}
\NormalTok{))}
\end{Highlighting}
\end{Shaded}

\begin{verbatim}
Quantidade de municípios a receberem qualquer tipo de emenda: 666
Quantidade de municípios a receberem emendas individuais: 633
Quantidade de municípios a receberem emendas de bancada: 66
Quantidade de municípios a receberem emendas de relator: 1
\end{verbatim}

\begin{itemize}
\tightlist
\item
  Note que 666 municípios brasileiros foram favorecidos por algum
  empenho de emenda em 2021 e que essas emendas foram somente
  individuais, de bancada ou de relator.
\end{itemize}

\subsection{1.2) Junção de dados}\label{junuxe7uxe3o-de-dados}

\subsubsection{1.2.1) Juntar bases de municípios e
eleições}\label{juntar-bases-de-municuxedpios-e-eleiuxe7uxf5es}

\begin{Shaded}
\begin{Highlighting}[]
\CommentTok{\# Ver se algum município de dt\_eleicoes não está em dt\_municípios}
\NormalTok{municipios\_dt\_eleicoes }\OtherTok{\textless{}{-}} \FunctionTok{unique}\NormalTok{(dt\_eleicoes[, }\FunctionTok{paste0}\NormalTok{(id\_municipio\_nome, sigla\_uf)])}
\NormalTok{municipios\_dt\_municipios }\OtherTok{\textless{}{-}} \FunctionTok{unique}\NormalTok{(dt\_municipios[, }\FunctionTok{paste0}\NormalTok{(id\_municipio\_nome, sigla\_uf)])}
\ControlFlowTok{for}\NormalTok{ (muni }\ControlFlowTok{in}\NormalTok{ municipios\_dt\_eleicoes) \{}
  \ControlFlowTok{if}\NormalTok{ (}\SpecialCharTok{!}\NormalTok{muni }\SpecialCharTok{\%in\%}\NormalTok{ municipios\_dt\_municipios) \{}
    \FunctionTok{print}\NormalTok{(muni)}
\NormalTok{  \}}
\NormalTok{\}}
\end{Highlighting}
\end{Shaded}

\begin{verbatim}
[1] "Restinga SecaRS"
[1] "São Vicente FerrerPE"
[1] "Lauro MullerSC"
[1] "Santa TeresinhaBA"
[1] "Grão ParáSC"
[1] "São Cristovão do SulSC"
[1] "São Thomé das LetrasMG"
[1] "Amparo de São FranciscoSE"
[1] "WestfaliaRS"
[1] "Vespasiano CorreaRS"
\end{verbatim}

\begin{Shaded}
\begin{Highlighting}[]
\FunctionTok{rm}\NormalTok{(municipios\_dt\_eleicoes, municipios\_dt\_municipios, muni)}

\CommentTok{\# Renomear municípios de dt\_municipios fora do padrão de dt\_eleicoes}
\NormalTok{dt\_municipios[id\_municipio\_nome }\SpecialCharTok{==} \StringTok{"Santa Terezinha"} \SpecialCharTok{\&}\NormalTok{ sigla\_uf }\SpecialCharTok{==} \StringTok{"BA"}\NormalTok{,}
\NormalTok{              id\_municipio\_nome }\SpecialCharTok{:}\ErrorTok{=} \StringTok{"Santa Teresinha"}\NormalTok{]}
\NormalTok{dt\_municipios[id\_municipio\_nome }\SpecialCharTok{==} \StringTok{"São Tomé das Letras"} \SpecialCharTok{\&}\NormalTok{ sigla\_uf }\SpecialCharTok{==} \StringTok{"MG"}\NormalTok{,}
\NormalTok{              id\_municipio\_nome }\SpecialCharTok{:}\ErrorTok{=} \StringTok{"São Thomé das Letras"}\NormalTok{]}
\NormalTok{dt\_municipios[id\_municipio\_nome }\SpecialCharTok{==} \StringTok{"São Vicente Férrer"} \SpecialCharTok{\&}\NormalTok{ sigla\_uf }\SpecialCharTok{==} \StringTok{"PE"}\NormalTok{,}
\NormalTok{              id\_municipio\_nome }\SpecialCharTok{:}\ErrorTok{=} \StringTok{"São Vicente Ferrer"}\NormalTok{]}
\NormalTok{dt\_municipios[id\_municipio\_nome }\SpecialCharTok{==} \StringTok{"Restinga Sêca"} \SpecialCharTok{\&}\NormalTok{ sigla\_uf }\SpecialCharTok{==} \StringTok{"RS"}\NormalTok{,}
\NormalTok{              id\_municipio\_nome }\SpecialCharTok{:}\ErrorTok{=} \StringTok{"Restinga Seca"}\NormalTok{]}
\NormalTok{dt\_municipios[id\_municipio\_nome }\SpecialCharTok{==} \StringTok{"Vespasiano Corrêa"} \SpecialCharTok{\&}\NormalTok{ sigla\_uf }\SpecialCharTok{==} \StringTok{"RS"}\NormalTok{,}
\NormalTok{              id\_municipio\_nome }\SpecialCharTok{:}\ErrorTok{=} \StringTok{"Vespasiano Correa"}\NormalTok{]}
\NormalTok{dt\_municipios[id\_municipio\_nome }\SpecialCharTok{==} \StringTok{"Westfália"} \SpecialCharTok{\&}\NormalTok{ sigla\_uf }\SpecialCharTok{==} \StringTok{"RS"}\NormalTok{,}
\NormalTok{              id\_municipio\_nome }\SpecialCharTok{:}\ErrorTok{=} \StringTok{"Westfalia"}\NormalTok{]}
\NormalTok{dt\_municipios[id\_municipio\_nome }\SpecialCharTok{==} \StringTok{"Grão{-}Pará"} \SpecialCharTok{\&}\NormalTok{ sigla\_uf }\SpecialCharTok{==} \StringTok{"SC"}\NormalTok{,}
\NormalTok{              id\_municipio\_nome }\SpecialCharTok{:}\ErrorTok{=} \StringTok{"Grão Pará"}\NormalTok{]}
\NormalTok{dt\_municipios[id\_municipio\_nome }\SpecialCharTok{==} \StringTok{"Lauro Müller"} \SpecialCharTok{\&}\NormalTok{ sigla\_uf }\SpecialCharTok{==} \StringTok{"SC"}\NormalTok{,}
\NormalTok{              id\_municipio\_nome }\SpecialCharTok{:}\ErrorTok{=} \StringTok{"Lauro Muller"}\NormalTok{]}
\NormalTok{dt\_municipios[id\_municipio\_nome }\SpecialCharTok{==} \StringTok{"São Cristóvão do Sul"} \SpecialCharTok{\&}\NormalTok{ sigla\_uf }\SpecialCharTok{==} \StringTok{"SC"}\NormalTok{,}
\NormalTok{              id\_municipio\_nome }\SpecialCharTok{:}\ErrorTok{=} \StringTok{"São Cristovão do Sul"}\NormalTok{]}
\NormalTok{dt\_municipios[id\_municipio\_nome }\SpecialCharTok{==} \StringTok{"Amparo do São Francisco"} \SpecialCharTok{\&}\NormalTok{ sigla\_uf }\SpecialCharTok{==} \StringTok{"SE"}\NormalTok{,}
\NormalTok{              id\_municipio\_nome }\SpecialCharTok{:}\ErrorTok{=} \StringTok{"Amparo de São Francisco"}\NormalTok{]}

\CommentTok{\# Repetir procedimento para ver se deu certo}
\NormalTok{municipios\_dt\_eleicoes }\OtherTok{\textless{}{-}} \FunctionTok{unique}\NormalTok{(dt\_eleicoes[, }\FunctionTok{paste0}\NormalTok{(id\_municipio\_nome, sigla\_uf)])}
\NormalTok{municipios\_dt\_municipios }\OtherTok{\textless{}{-}} \FunctionTok{unique}\NormalTok{(dt\_municipios[, }\FunctionTok{paste0}\NormalTok{(id\_municipio\_nome, sigla\_uf)])}
\ControlFlowTok{for}\NormalTok{ (muni }\ControlFlowTok{in}\NormalTok{ municipios\_dt\_eleicoes) \{}
  \ControlFlowTok{if}\NormalTok{ (}\SpecialCharTok{!}\NormalTok{muni }\SpecialCharTok{\%in\%}\NormalTok{ municipios\_dt\_municipios) \{}
    \FunctionTok{print}\NormalTok{(muni)}
\NormalTok{  \}}
\NormalTok{\}}
\FunctionTok{rm}\NormalTok{(municipios\_dt\_eleicoes, municipios\_dt\_municipios, muni)}

\CommentTok{\# Adicionar dados sobre os municípios com eleições de interesse}
\NormalTok{dt\_municipios\_eleicoes }\OtherTok{\textless{}{-}} \FunctionTok{merge}\NormalTok{(dt\_eleicoes, dt\_municipios, }\AttributeTok{all.x =}\NormalTok{ T)}
\FunctionTok{rm}\NormalTok{(dt\_eleicoes, dt\_municipios)}

\CommentTok{\# Dar uma olhada na base de dados resultante}
\FunctionTok{str}\NormalTok{(dt\_municipios\_eleicoes)}
\end{Highlighting}
\end{Shaded}

\begin{verbatim}
Classes 'data.table' and 'data.frame':  1196 obs. of  13 variables:
 $ id_municipio_nome      : chr  "Abadia de Goiás" "Abadiânia" "Abel Figueiredo" "Acrelândia" ...
 $ sigla_uf               : chr  "GO" "GO" "PA" "AC" ...
 $ id_municipio           : chr  "5200050" "5200100" "1500131" "1200013" ...
 $ vitoria_pp             : num  1 1 0 0 1 0 1 1 0 1 ...
 $ votos_razao_pp         : num  0.694 0.601 0.491 0.477 0.502 ...
 $ votos_razao_pp_centr   : num  0.19419 0.10064 -0.00885 -0.0231 0.00169 ...
 $ votos_razao_pp_centr_sq: num  3.77e-02 1.01e-02 7.83e-05 5.34e-04 2.85e-06 ...
 $ mortalidade_infantil   : num  13.94 16.13 NA NA 7.52 ...
 $ PIBpc                  : num  38622 21335 13999 25363 46316 ...
 $ taxa_escolarizacao     : num  0.945 0.99 0.997 0.961 0.982 ...
 $ pop                    : num  19128 17232 6136 14021 21568 ...
 $ densidade_demografica  : num  133.43 16.5 9.99 7.74 13.77 ...
 $ regiao                 : chr  "Centro-Oeste" "Centro-Oeste" "Norte" "Norte" ...
 - attr(*, ".internal.selfref")=<externalptr> 
 - attr(*, "sorted")= chr [1:2] "id_municipio_nome" "sigla_uf"
\end{verbatim}

\begin{Shaded}
\begin{Highlighting}[]
\CommentTok{\# Verificar se houve algum NA}
\FunctionTok{colSums}\NormalTok{(}\FunctionTok{is.na}\NormalTok{(dt\_municipios\_eleicoes))}
\end{Highlighting}
\end{Shaded}

\begin{verbatim}
      id_municipio_nome                sigla_uf            id_municipio 
                      0                       0                       0 
             vitoria_pp          votos_razao_pp    votos_razao_pp_centr 
                      0                       0                       0 
votos_razao_pp_centr_sq    mortalidade_infantil                   PIBpc 
                      0                     362                       0 
     taxa_escolarizacao                     pop   densidade_demografica 
                      0                       0                       0 
                 regiao 
                      0 
\end{verbatim}

\begin{itemize}
\tightlist
\item
  Como mortalidade infantil tem vários valores NA, não vamos analisá-la
  nem usá-la como controle.
\end{itemize}

\subsubsection{1.2.2) Juntar base de municípios e eleições com bases de
emendas}\label{juntar-base-de-municuxedpios-e-eleiuxe7uxf5es-com-bases-de-emendas}

\paragraph{1.2.2.1) Para emendas de todos os
tipos}\label{para-emendas-de-todos-os-tipos}

\begin{Shaded}
\begin{Highlighting}[]
\CommentTok{\# Resumir emendas por id\_municipio, independentemente de tipo}
\NormalTok{dt\_emendas\_todas }\OtherTok{\textless{}{-}}\NormalTok{ dt\_emendas[,}
\NormalTok{                               .(}\AttributeTok{valor\_empenhado\_ipca =} \FunctionTok{sum}\NormalTok{(valor\_empenhado\_ipca)),}
\NormalTok{                               by }\OtherTok{=}\NormalTok{ id\_municipio]}

\CommentTok{\# Adicionar dados sobre emendas ao dt\_municipios\_eleicoes}
\NormalTok{dt\_todas }\OtherTok{\textless{}{-}} \FunctionTok{merge}\NormalTok{(dt\_municipios\_eleicoes, dt\_emendas\_todas, }\AttributeTok{all.x =}\NormalTok{ T, }\AttributeTok{by =} \StringTok{"id\_municipio"}\NormalTok{)}
\FunctionTok{rm}\NormalTok{(dt\_emendas\_todas)}

\CommentTok{\# Transformar NAs em 0}
\NormalTok{dt\_todas[}\FunctionTok{is.na}\NormalTok{(valor\_empenhado\_ipca), valor\_empenhado\_ipca }\SpecialCharTok{:}\ErrorTok{=} \DecValTok{0}\NormalTok{]}

\CommentTok{\# Adicionar coluna de valor empenhado per capita (usando valor de 2021 e pop de 2022)}
\NormalTok{dt\_todas[, valor\_empenhado\_ipca\_pc }\SpecialCharTok{:}\ErrorTok{=}\NormalTok{ valor\_empenhado\_ipca}\SpecialCharTok{/}\NormalTok{pop]}
\end{Highlighting}
\end{Shaded}

\paragraph{1.2.2.2) Para emendas
individuais}\label{para-emendas-individuais}

\begin{Shaded}
\begin{Highlighting}[]
\CommentTok{\# Filtrar por somente emendas individuais}
\NormalTok{dt\_emendas\_individuais }\OtherTok{\textless{}{-}}\NormalTok{ dt\_emendas[tipo\_emenda }\SpecialCharTok{==} \StringTok{"individual"}\NormalTok{]}
\CommentTok{\# Remover coluna de tipo}
\NormalTok{dt\_emendas\_individuais[, tipo\_emenda }\SpecialCharTok{:}\ErrorTok{=} \ConstantTok{NULL}\NormalTok{]}

\CommentTok{\# Adicionar dados sobre emendas ao dt\_municipios\_eleicoes}
\NormalTok{dt\_individuais }\OtherTok{\textless{}{-}} \FunctionTok{merge}\NormalTok{(dt\_municipios\_eleicoes, dt\_emendas\_individuais, }\AttributeTok{all.x =}\NormalTok{ T, }\AttributeTok{by =} \StringTok{"id\_municipio"}\NormalTok{)}
\FunctionTok{rm}\NormalTok{(dt\_emendas\_individuais)}

\CommentTok{\# Transformar NAs em 0}
\NormalTok{dt\_individuais[}\FunctionTok{is.na}\NormalTok{(valor\_empenhado\_ipca), valor\_empenhado\_ipca }\SpecialCharTok{:}\ErrorTok{=} \DecValTok{0}\NormalTok{]}

\CommentTok{\# Adicionar coluna de valor empenhado per capita (usando valor de 2021 e pop de 2022)}
\NormalTok{dt\_individuais[, valor\_empenhado\_ipca\_pc }\SpecialCharTok{:}\ErrorTok{=}\NormalTok{ valor\_empenhado\_ipca}\SpecialCharTok{/}\NormalTok{pop]}
\end{Highlighting}
\end{Shaded}

\paragraph{1.2.2.3) Para emendas de
bancada}\label{para-emendas-de-bancada}

\begin{Shaded}
\begin{Highlighting}[]
\CommentTok{\# Filtrar por somente emendas de bancada}
\NormalTok{dt\_emendas\_bancada }\OtherTok{\textless{}{-}}\NormalTok{ dt\_emendas[tipo\_emenda }\SpecialCharTok{==} \StringTok{"bancada"}\NormalTok{]}
\CommentTok{\# Remover coluna de tipo}
\NormalTok{dt\_emendas\_bancada[, tipo\_emenda }\SpecialCharTok{:}\ErrorTok{=} \ConstantTok{NULL}\NormalTok{]}

\CommentTok{\# Adicionar dados sobre emendas ao dt\_municipios\_eleicoes}
\NormalTok{dt\_bancada }\OtherTok{\textless{}{-}} \FunctionTok{merge}\NormalTok{(dt\_municipios\_eleicoes, dt\_emendas\_bancada, }\AttributeTok{all.x =}\NormalTok{ T, }\AttributeTok{by =} \StringTok{"id\_municipio"}\NormalTok{)}
\FunctionTok{rm}\NormalTok{(dt\_emendas\_bancada)}

\CommentTok{\# Transformar NAs em 0}
\NormalTok{dt\_bancada[}\FunctionTok{is.na}\NormalTok{(valor\_empenhado\_ipca), valor\_empenhado\_ipca }\SpecialCharTok{:}\ErrorTok{=} \DecValTok{0}\NormalTok{]}

\CommentTok{\# Adicionar coluna de valor empenhado per capita (usando valor de 2021 e pop de 2022)}
\NormalTok{dt\_bancada[, valor\_empenhado\_ipca\_pc }\SpecialCharTok{:}\ErrorTok{=}\NormalTok{ valor\_empenhado\_ipca}\SpecialCharTok{/}\NormalTok{pop]}
\end{Highlighting}
\end{Shaded}

\paragraph{1.2.2.4) Para emendas de
relator}\label{para-emendas-de-relator}

\begin{Shaded}
\begin{Highlighting}[]
\CommentTok{\# Filtrar por somente emendas de relator}
\NormalTok{dt\_emendas\_relator }\OtherTok{\textless{}{-}}\NormalTok{ dt\_emendas[tipo\_emenda }\SpecialCharTok{==} \StringTok{"relator"}\NormalTok{]}
\CommentTok{\# Remover coluna de tipo}
\NormalTok{dt\_emendas\_relator[, tipo\_emenda }\SpecialCharTok{:}\ErrorTok{=} \ConstantTok{NULL}\NormalTok{]}

\CommentTok{\# Adicionar dados sobre emendas ao dt\_municipios\_eleicoes}
\NormalTok{dt\_relator }\OtherTok{\textless{}{-}} \FunctionTok{merge}\NormalTok{(dt\_municipios\_eleicoes, dt\_emendas\_relator, }\AttributeTok{all.x =}\NormalTok{ T, }\AttributeTok{by =} \StringTok{"id\_municipio"}\NormalTok{)}
\FunctionTok{rm}\NormalTok{(dt\_emendas\_relator)}

\CommentTok{\# Transformar NAs em 0}
\NormalTok{dt\_relator[}\FunctionTok{is.na}\NormalTok{(valor\_empenhado\_ipca), valor\_empenhado\_ipca }\SpecialCharTok{:}\ErrorTok{=} \DecValTok{0}\NormalTok{]}

\CommentTok{\# Adicionar coluna de valor empenhado per capita (usando valor de 2021 e pop de 2022)}
\NormalTok{dt\_relator[, valor\_empenhado\_ipca\_pc }\SpecialCharTok{:}\ErrorTok{=}\NormalTok{ valor\_empenhado\_ipca}\SpecialCharTok{/}\NormalTok{pop]}
\end{Highlighting}
\end{Shaded}

\begin{Shaded}
\begin{Highlighting}[]
\FunctionTok{rm}\NormalTok{(dt\_emendas)}
\end{Highlighting}
\end{Shaded}

\section{2) Estatísticas descritivas}\label{estatuxedsticas-descritivas}

\subsection{2.1) Verificação de
(des)continuidades}\label{verificauxe7uxe3o-de-descontinuidades}

Note que filtramos os municípios para os percentis 2 a 99 de cada
característica para evitar distorções causadas por outliers.

\subsubsection{2.1.1) Características
municipais}\label{caracteruxedsticas-municipais-1}

\begin{Shaded}
\begin{Highlighting}[]
\CommentTok{\# Linear}
\NormalTok{g\_pop }\OtherTok{\textless{}{-}}\NormalTok{ dt\_todas[pop }\SpecialCharTok{\%between\%} \FunctionTok{quantile}\NormalTok{(pop, }\FunctionTok{c}\NormalTok{(.}\DecValTok{02}\NormalTok{, .}\DecValTok{99}\NormalTok{), }\AttributeTok{na.rm =} \ConstantTok{TRUE}\NormalTok{)] }\SpecialCharTok{\%\textgreater{}\%}
  \FunctionTok{ggplot}\NormalTok{(}\FunctionTok{aes}\NormalTok{(}\AttributeTok{x =}\NormalTok{ votos\_razao\_pp\_centr, }\AttributeTok{y =}\NormalTok{ pop, }\AttributeTok{color =} \FunctionTok{as.factor}\NormalTok{(vitoria\_pp))) }\SpecialCharTok{+}
  \FunctionTok{geom\_point}\NormalTok{(}\AttributeTok{shape =} \DecValTok{21}\NormalTok{, }\AttributeTok{size =} \FloatTok{0.5}\NormalTok{, }\AttributeTok{color =} \StringTok{"black"}\NormalTok{) }\SpecialCharTok{+}
  \FunctionTok{stat\_smooth}\NormalTok{(}\AttributeTok{method =} \StringTok{"lm"}\NormalTok{, }\AttributeTok{se =}\NormalTok{ F) }\SpecialCharTok{+}
  \FunctionTok{scale\_color\_brewer}\NormalTok{(}\AttributeTok{palette =} \StringTok{"Dark2"}\NormalTok{, }\AttributeTok{direction =} \SpecialCharTok{{-}}\DecValTok{1}\NormalTok{) }\SpecialCharTok{+}
  \FunctionTok{geom\_vline}\NormalTok{(}\AttributeTok{xintercept =} \FloatTok{0.5}\NormalTok{, }\AttributeTok{linetype =} \StringTok{"dashed"}\NormalTok{) }\SpecialCharTok{+}
  \FunctionTok{scale\_y\_continuous}\NormalTok{(}\AttributeTok{labels =} \FunctionTok{label\_number}\NormalTok{()) }\SpecialCharTok{+}
  \FunctionTok{theme\_minimal}\NormalTok{() }\SpecialCharTok{+} \FunctionTok{guides}\NormalTok{(}\AttributeTok{color =} \StringTok{"none"}\NormalTok{) }\SpecialCharTok{+}
  \FunctionTok{labs}\NormalTok{(}\AttributeTok{title =} \StringTok{""}\NormalTok{,}
       \AttributeTok{y =} \StringTok{"População"}\NormalTok{,}
       \AttributeTok{x =} \StringTok{"Margem de vitória do PP"}\NormalTok{) }\SpecialCharTok{+} \FunctionTok{scale\_x\_continuous}\NormalTok{(}\AttributeTok{labels =} \FunctionTok{label\_percent}\NormalTok{()) }\SpecialCharTok{+} \FunctionTok{scale\_x\_continuous}\NormalTok{(}\AttributeTok{labels =} \FunctionTok{label\_percent}\NormalTok{())}

\NormalTok{g\_densidade\_demografica }\OtherTok{\textless{}{-}}\NormalTok{ dt\_todas[densidade\_demografica }\SpecialCharTok{\%between\%} \FunctionTok{quantile}\NormalTok{(densidade\_demografica, }\FunctionTok{c}\NormalTok{(.}\DecValTok{02}\NormalTok{, .}\DecValTok{99}\NormalTok{), }\AttributeTok{na.rm =} \ConstantTok{TRUE}\NormalTok{)] }\SpecialCharTok{\%\textgreater{}\%}
  \FunctionTok{ggplot}\NormalTok{(}\FunctionTok{aes}\NormalTok{(}\AttributeTok{x =}\NormalTok{ votos\_razao\_pp\_centr, }\AttributeTok{y =}\NormalTok{ densidade\_demografica, }\AttributeTok{color =} \FunctionTok{as.factor}\NormalTok{(vitoria\_pp))) }\SpecialCharTok{+}
  \FunctionTok{geom\_point}\NormalTok{(}\AttributeTok{shape =} \DecValTok{21}\NormalTok{, }\AttributeTok{size =} \FloatTok{0.5}\NormalTok{, }\AttributeTok{color =} \StringTok{"black"}\NormalTok{) }\SpecialCharTok{+}
  \FunctionTok{stat\_smooth}\NormalTok{(}\AttributeTok{method =} \StringTok{"lm"}\NormalTok{, }\AttributeTok{se =}\NormalTok{ F) }\SpecialCharTok{+}
  \FunctionTok{scale\_color\_brewer}\NormalTok{(}\AttributeTok{palette =} \StringTok{"Dark2"}\NormalTok{, }\AttributeTok{direction =} \SpecialCharTok{{-}}\DecValTok{1}\NormalTok{) }\SpecialCharTok{+}
  \FunctionTok{geom\_vline}\NormalTok{(}\AttributeTok{xintercept =} \FloatTok{0.5}\NormalTok{, }\AttributeTok{linetype =} \StringTok{"dashed"}\NormalTok{) }\SpecialCharTok{+}
  \FunctionTok{scale\_y\_continuous}\NormalTok{(}\AttributeTok{labels =} \FunctionTok{label\_number}\NormalTok{()) }\SpecialCharTok{+}
  \FunctionTok{theme\_minimal}\NormalTok{() }\SpecialCharTok{+} \FunctionTok{guides}\NormalTok{(}\AttributeTok{color =} \StringTok{"none"}\NormalTok{) }\SpecialCharTok{+}
  \FunctionTok{labs}\NormalTok{(}\AttributeTok{title =} \StringTok{""}\NormalTok{,}
       \AttributeTok{y =} \StringTok{"Densidade}\SpecialCharTok{\textbackslash{}n}\StringTok{demográfica"}\NormalTok{,}
       \AttributeTok{x =} \StringTok{"Margem de vitória do PP"}\NormalTok{) }\SpecialCharTok{+} \FunctionTok{scale\_x\_continuous}\NormalTok{(}\AttributeTok{labels =} \FunctionTok{label\_percent}\NormalTok{())}

\NormalTok{g\_PIBpc }\OtherTok{\textless{}{-}}\NormalTok{ dt\_todas[PIBpc }\SpecialCharTok{\%between\%} \FunctionTok{quantile}\NormalTok{(PIBpc, }\FunctionTok{c}\NormalTok{(.}\DecValTok{02}\NormalTok{, .}\DecValTok{99}\NormalTok{), }\AttributeTok{na.rm =} \ConstantTok{TRUE}\NormalTok{)] }\SpecialCharTok{\%\textgreater{}\%}
  \FunctionTok{ggplot}\NormalTok{(}\FunctionTok{aes}\NormalTok{(}\AttributeTok{x =}\NormalTok{ votos\_razao\_pp\_centr, }\AttributeTok{y =}\NormalTok{ PIBpc, }\AttributeTok{color =} \FunctionTok{as.factor}\NormalTok{(vitoria\_pp))) }\SpecialCharTok{+}
  \FunctionTok{geom\_point}\NormalTok{(}\AttributeTok{shape =} \DecValTok{21}\NormalTok{, }\AttributeTok{size =} \FloatTok{0.5}\NormalTok{, }\AttributeTok{color =} \StringTok{"black"}\NormalTok{) }\SpecialCharTok{+}
  \FunctionTok{stat\_smooth}\NormalTok{(}\AttributeTok{method =} \StringTok{"lm"}\NormalTok{, }\AttributeTok{se =}\NormalTok{ F) }\SpecialCharTok{+}
  \FunctionTok{scale\_color\_brewer}\NormalTok{(}\AttributeTok{palette =} \StringTok{"Dark2"}\NormalTok{, }\AttributeTok{direction =} \SpecialCharTok{{-}}\DecValTok{1}\NormalTok{) }\SpecialCharTok{+}
  \FunctionTok{geom\_vline}\NormalTok{(}\AttributeTok{xintercept =} \FloatTok{0.5}\NormalTok{, }\AttributeTok{linetype =} \StringTok{"dashed"}\NormalTok{) }\SpecialCharTok{+}
  \FunctionTok{scale\_y\_continuous}\NormalTok{(}\AttributeTok{labels =} \FunctionTok{label\_number}\NormalTok{()) }\SpecialCharTok{+}
  \FunctionTok{theme\_minimal}\NormalTok{() }\SpecialCharTok{+} \FunctionTok{guides}\NormalTok{(}\AttributeTok{color =} \StringTok{"none"}\NormalTok{) }\SpecialCharTok{+}
  \FunctionTok{labs}\NormalTok{(}\AttributeTok{title =} \StringTok{""}\NormalTok{,}
       \AttributeTok{y =} \StringTok{"PIB per capita"}\NormalTok{,}
       \AttributeTok{x =} \StringTok{"Margem de vitória do PP"}\NormalTok{) }\SpecialCharTok{+} \FunctionTok{scale\_x\_continuous}\NormalTok{(}\AttributeTok{labels =} \FunctionTok{label\_percent}\NormalTok{())}

\NormalTok{g\_taxa\_escolarizacao }\OtherTok{\textless{}{-}}\NormalTok{ dt\_todas[taxa\_escolarizacao }\SpecialCharTok{\%between\%} \FunctionTok{quantile}\NormalTok{(taxa\_escolarizacao, }\FunctionTok{c}\NormalTok{(.}\DecValTok{02}\NormalTok{, .}\DecValTok{99}\NormalTok{), }\AttributeTok{na.rm =} \ConstantTok{TRUE}\NormalTok{)] }\SpecialCharTok{\%\textgreater{}\%}
  \FunctionTok{ggplot}\NormalTok{(}\FunctionTok{aes}\NormalTok{(}\AttributeTok{x =}\NormalTok{ votos\_razao\_pp\_centr, }\AttributeTok{y =}\NormalTok{ taxa\_escolarizacao, }\AttributeTok{color =} \FunctionTok{as.factor}\NormalTok{(vitoria\_pp))) }\SpecialCharTok{+}
  \FunctionTok{geom\_point}\NormalTok{(}\AttributeTok{shape =} \DecValTok{21}\NormalTok{, }\AttributeTok{size =} \FloatTok{0.5}\NormalTok{, }\AttributeTok{color =} \StringTok{"black"}\NormalTok{) }\SpecialCharTok{+}
  \FunctionTok{stat\_smooth}\NormalTok{(}\AttributeTok{method =} \StringTok{"lm"}\NormalTok{, }\AttributeTok{se =}\NormalTok{ F) }\SpecialCharTok{+}
  \FunctionTok{scale\_color\_brewer}\NormalTok{(}\AttributeTok{palette =} \StringTok{"Dark2"}\NormalTok{, }\AttributeTok{direction =} \SpecialCharTok{{-}}\DecValTok{1}\NormalTok{) }\SpecialCharTok{+}
  \FunctionTok{geom\_vline}\NormalTok{(}\AttributeTok{xintercept =} \FloatTok{0.5}\NormalTok{, }\AttributeTok{linetype =} \StringTok{"dashed"}\NormalTok{) }\SpecialCharTok{+}
  \FunctionTok{scale\_y\_continuous}\NormalTok{(}\AttributeTok{labels =} \FunctionTok{label\_number}\NormalTok{()) }\SpecialCharTok{+}
  \FunctionTok{theme\_minimal}\NormalTok{() }\SpecialCharTok{+} \FunctionTok{guides}\NormalTok{(}\AttributeTok{color =} \StringTok{"none"}\NormalTok{) }\SpecialCharTok{+}
  \FunctionTok{labs}\NormalTok{(}\AttributeTok{title =} \StringTok{""}\NormalTok{,}
       \AttributeTok{y =} \StringTok{"Taxa de}\SpecialCharTok{\textbackslash{}n}\StringTok{escolarização"}\NormalTok{,}
       \AttributeTok{x =} \StringTok{"Margem de vitória do PP"}\NormalTok{) }\SpecialCharTok{+} \FunctionTok{scale\_x\_continuous}\NormalTok{(}\AttributeTok{labels =} \FunctionTok{label\_percent}\NormalTok{())}

\FunctionTok{plot\_grid}\NormalTok{(g\_pop, g\_densidade\_demografica, g\_PIBpc, g\_taxa\_escolarizacao, }\AttributeTok{ncol =} \DecValTok{2}\NormalTok{)}
\end{Highlighting}
\end{Shaded}

\includegraphics{estimacao_principal_brasil_20251123_files/figure-pdf/grafico_caracteristicas_municipais_por_razao_votos_pp-1.pdf}

\subsubsection{\texorpdfstring{2.1.2) Valor empenhado \emph{per capita}
de emendas
parlamentares}{2.1.2) Valor empenhado per capita de emendas parlamentares}}\label{valor-empenhado-per-capita-de-emendas-parlamentares}

\begin{Shaded}
\begin{Highlighting}[]
\CommentTok{\# Linear}
\NormalTok{dt\_individuais[valor\_empenhado\_ipca\_pc }\SpecialCharTok{\%between\%} \FunctionTok{quantile}\NormalTok{(valor\_empenhado\_ipca\_pc, }\FunctionTok{c}\NormalTok{(.}\DecValTok{02}\NormalTok{, .}\DecValTok{99}\NormalTok{), }\AttributeTok{na.rm =} \ConstantTok{TRUE}\NormalTok{)] }\SpecialCharTok{\%\textgreater{}\%}
  \FunctionTok{ggplot}\NormalTok{(}\FunctionTok{aes}\NormalTok{(}\AttributeTok{x =}\NormalTok{ votos\_razao\_pp\_centr, }\AttributeTok{y =}\NormalTok{ valor\_empenhado\_ipca\_pc, }\AttributeTok{color =} \FunctionTok{as.factor}\NormalTok{(vitoria\_pp))) }\SpecialCharTok{+}
  \FunctionTok{geom\_point}\NormalTok{(}\AttributeTok{shape =} \DecValTok{21}\NormalTok{, }\AttributeTok{color =} \StringTok{"black"}\NormalTok{) }\SpecialCharTok{+}
  \FunctionTok{stat\_smooth}\NormalTok{(}\AttributeTok{method =} \StringTok{"lm"}\NormalTok{, }\AttributeTok{formula =}\NormalTok{ y }\SpecialCharTok{\textasciitilde{}}\NormalTok{ x, }\AttributeTok{se =}\NormalTok{ F) }\SpecialCharTok{+}
  \FunctionTok{scale\_color\_brewer}\NormalTok{(}\AttributeTok{palette =} \StringTok{"Dark2"}\NormalTok{, }\AttributeTok{direction =} \SpecialCharTok{{-}}\DecValTok{1}\NormalTok{) }\SpecialCharTok{+}
  \FunctionTok{geom\_vline}\NormalTok{(}\AttributeTok{xintercept =} \FloatTok{0.5}\NormalTok{, }\AttributeTok{linetype =} \StringTok{"dashed"}\NormalTok{) }\SpecialCharTok{+}
  \FunctionTok{scale\_y\_continuous}\NormalTok{(}\AttributeTok{labels =} \FunctionTok{label\_number}\NormalTok{()) }\SpecialCharTok{+}
  \FunctionTok{theme\_minimal}\NormalTok{() }\SpecialCharTok{+} \FunctionTok{guides}\NormalTok{(}\AttributeTok{color =} \StringTok{"none"}\NormalTok{) }\SpecialCharTok{+}
  \FunctionTok{labs}\NormalTok{(}\AttributeTok{title =} \StringTok{""}\NormalTok{,}
       \AttributeTok{y =} \StringTok{"Valor empenhado per capita"}\NormalTok{,}
       \AttributeTok{x =} \StringTok{"Margem de vitória do PP"}\NormalTok{) }\SpecialCharTok{+} \FunctionTok{scale\_x\_continuous}\NormalTok{(}\AttributeTok{labels =} \FunctionTok{label\_percent}\NormalTok{())}
\end{Highlighting}
\end{Shaded}

\includegraphics{estimacao_principal_brasil_20251123_files/figure-pdf/grafico_valor_empenhado_por_razao_votos_pp-1.pdf}

\subsection{2.2) Comparação entre tratados e
controles}\label{comparauxe7uxe3o-entre-tratados-e-controles}

\subsubsection{2.2.0) Obter janela ótima e filtrar bases de
dados}\label{obter-janela-uxf3tima-e-filtrar-bases-de-dados}

Agora, vamos mostrar que os municípios em que o candidato do PP ganhou
vs.~perdeu por pouco de fato são semelhantes. Para isso, vamos filtrar
as bases de dados por eleições acirradas.

\begin{Shaded}
\begin{Highlighting}[]
\CommentTok{\# Guardar bases de dados completas}
\NormalTok{dt\_todas\_completa }\OtherTok{\textless{}{-}}\NormalTok{ dt\_todas}
\NormalTok{dt\_individuais\_completa }\OtherTok{\textless{}{-}}\NormalTok{ dt\_individuais}
\NormalTok{dt\_bancada\_completa }\OtherTok{\textless{}{-}}\NormalTok{ dt\_bancada}
\NormalTok{dt\_relator\_completa }\OtherTok{\textless{}{-}}\NormalTok{ dt\_relator}

\CommentTok{\# Obter janela ótima para modelo linear com covariadas}
\NormalTok{covariadas }\OtherTok{\textless{}{-}}\NormalTok{ dt\_individuais[, .(pop, densidade\_demografica, PIBpc, taxa\_escolarizacao)]}
\NormalTok{janela }\OtherTok{\textless{}{-}} \FunctionTok{rdbwselect}\NormalTok{(}
  \AttributeTok{x =}\NormalTok{ dt\_individuais}\SpecialCharTok{$}\NormalTok{votos\_razao\_pp\_centr,}
  \AttributeTok{y =}\NormalTok{ dt\_individuais}\SpecialCharTok{$}\NormalTok{valor\_empenhado\_ipca\_pc,}
  \AttributeTok{p =} \DecValTok{1}\NormalTok{,}
  \AttributeTok{covs =}\NormalTok{ covariadas}
\NormalTok{)}
\NormalTok{janela}\SpecialCharTok{$}\NormalTok{bws}
\end{Highlighting}
\end{Shaded}

\begin{verbatim}
        h (left)  h (right)  b (left) b (right)
mserd 0.08415324 0.08415324 0.1634445 0.1634445
\end{verbatim}

\begin{Shaded}
\begin{Highlighting}[]
\CommentTok{\# Filtrar base de dados para municípios cuja margem de vitória do PP está na janela}
\NormalTok{dt\_todas }\OtherTok{\textless{}{-}}\NormalTok{ dt\_todas[votos\_razao\_pp\_centr }\SpecialCharTok{\textgreater{}=} \SpecialCharTok{{-}}\NormalTok{janela}\SpecialCharTok{$}\NormalTok{bws[}\DecValTok{1}\NormalTok{] }\SpecialCharTok{\&}
\NormalTok{                       votos\_razao\_pp\_centr }\SpecialCharTok{\textless{}=}\NormalTok{ janela}\SpecialCharTok{$}\NormalTok{bws[}\DecValTok{1}\NormalTok{]]}
\NormalTok{dt\_individuais }\OtherTok{\textless{}{-}}\NormalTok{ dt\_individuais[votos\_razao\_pp\_centr }\SpecialCharTok{\textgreater{}=} \SpecialCharTok{{-}}\NormalTok{janela}\SpecialCharTok{$}\NormalTok{bws[}\DecValTok{1}\NormalTok{] }\SpecialCharTok{\&}
\NormalTok{                                   votos\_razao\_pp\_centr }\SpecialCharTok{\textless{}=}\NormalTok{ janela}\SpecialCharTok{$}\NormalTok{bws[}\DecValTok{1}\NormalTok{]]}
\NormalTok{dt\_bancada }\OtherTok{\textless{}{-}}\NormalTok{ dt\_bancada[votos\_razao\_pp\_centr }\SpecialCharTok{\textgreater{}=} \SpecialCharTok{{-}}\NormalTok{janela}\SpecialCharTok{$}\NormalTok{bws[}\DecValTok{1}\NormalTok{] }\SpecialCharTok{\&}
\NormalTok{                           votos\_razao\_pp\_centr }\SpecialCharTok{\textless{}=}\NormalTok{ janela}\SpecialCharTok{$}\NormalTok{bws[}\DecValTok{1}\NormalTok{]]}
\NormalTok{dt\_relator }\OtherTok{\textless{}{-}}\NormalTok{ dt\_relator[votos\_razao\_pp\_centr }\SpecialCharTok{\textgreater{}=} \SpecialCharTok{{-}}\NormalTok{janela}\SpecialCharTok{$}\NormalTok{bws[}\DecValTok{1}\NormalTok{] }\SpecialCharTok{\&}
\NormalTok{                           votos\_razao\_pp\_centr }\SpecialCharTok{\textless{}=}\NormalTok{ janela}\SpecialCharTok{$}\NormalTok{bws[}\DecValTok{1}\NormalTok{]]}

\CommentTok{\# Ver quantos municípios com eleições acirradas receberam emendas}
\FunctionTok{writeLines}\NormalTok{(}\FunctionTok{paste0}\NormalTok{(}
  \StringTok{"Quantidade de municípios com eleições acirradas: "}\NormalTok{, }
  \FunctionTok{nrow}\NormalTok{(dt\_todas),}
  
  \StringTok{"}\SpecialCharTok{\textbackslash{}n}\StringTok{Quantidade de municípios com eleições acirradas que receberam emendas: "}\NormalTok{, }
  \FunctionTok{nrow}\NormalTok{(dt\_todas[valor\_empenhado\_ipca }\SpecialCharTok{!=} \DecValTok{0}\NormalTok{]),}

  \StringTok{"}\SpecialCharTok{\textbackslash{}n}\StringTok{Quantidade de municípios com eleições acirradas que receberam emendas individuais: "}\NormalTok{,}
  \FunctionTok{nrow}\NormalTok{(dt\_individuais[valor\_empenhado\_ipca }\SpecialCharTok{!=} \DecValTok{0}\NormalTok{]),}
  
  \StringTok{"}\SpecialCharTok{\textbackslash{}n}\StringTok{Quantidade de municípios com eleições acirradas que receberam emendas de bancada: "}\NormalTok{,}
  \FunctionTok{nrow}\NormalTok{(dt\_bancada[valor\_empenhado\_ipca }\SpecialCharTok{!=} \DecValTok{0}\NormalTok{]),}
  
  \StringTok{"}\SpecialCharTok{\textbackslash{}n}\StringTok{Quantidade de municípios com eleições acirradas que receberam emendas de relator: "}\NormalTok{,}
  \FunctionTok{nrow}\NormalTok{(dt\_relator[valor\_empenhado\_ipca }\SpecialCharTok{!=} \DecValTok{0}\NormalTok{])))}
\end{Highlighting}
\end{Shaded}

\begin{verbatim}
Quantidade de municípios com eleições acirradas: 708
Quantidade de municípios com eleições acirradas que receberam emendas: 62
Quantidade de municípios com eleições acirradas que receberam emendas individuais: 59
Quantidade de municípios com eleições acirradas que receberam emendas de bancada: 7
Quantidade de municípios com eleições acirradas que receberam emendas de relator: 0
\end{verbatim}

\begin{itemize}
\tightlist
\item
  Note que nenhum município com eleições acirradas recebeu emendas de
  relator e somente 7 receberam emendas de bancada, então vamos analisar
  somente emendas individuais.
\end{itemize}

\subsubsection{2.2.1) Características
municipais}\label{caracteruxedsticas-municipais-2}

\paragraph{2.2.1.1) Via médias}\label{via-muxe9dias}

\begin{Shaded}
\begin{Highlighting}[]
\CommentTok{\# Criar dt com médias}
\NormalTok{medias\_municipios\_eleicoes }\OtherTok{\textless{}{-}}\NormalTok{ dt\_municipios\_eleicoes[}
\NormalTok{  ,}
\NormalTok{  .(}
    \AttributeTok{media\_pop =} \FunctionTok{mean}\NormalTok{(pop),}
    \AttributeTok{media\_densidade\_demografica =} \FunctionTok{mean}\NormalTok{(densidade\_demografica),}
    \AttributeTok{media\_PIBpc =} \FunctionTok{mean}\NormalTok{(PIBpc),}
    \AttributeTok{media\_taxa\_escolarizacao =} \FunctionTok{mean}\NormalTok{(taxa\_escolarizacao),}
    \AttributeTok{media\_mortalidade\_infantil =} \FunctionTok{mean}\NormalTok{(mortalidade\_infantil, }\AttributeTok{na.rm =}\NormalTok{ T)}
\NormalTok{  ),}
\NormalTok{  by }\OtherTok{=}\NormalTok{ vitoria\_pp}
\NormalTok{]}

\CommentTok{\# Plotar médias de quem venceu vs. não venceu {-} formatar}
\NormalTok{titulo }\OtherTok{\textless{}{-}} \FunctionTok{ggplot}\NormalTok{() }\SpecialCharTok{+} 
  \FunctionTok{labs}\NormalTok{(}\AttributeTok{title =} \StringTok{"Comparação das médias de características municipais"}\NormalTok{, }\AttributeTok{subtitle =} \StringTok{"Fontes dos dados: TSE, IBGE (2021, 2022). Elaboração própria."}\NormalTok{) }\SpecialCharTok{+}
  \FunctionTok{theme\_minimal}\NormalTok{()}

\NormalTok{g1 }\OtherTok{\textless{}{-}}\NormalTok{ medias\_municipios\_eleicoes }\SpecialCharTok{\%\textgreater{}\%}
  \FunctionTok{ggplot}\NormalTok{(}\FunctionTok{aes}\NormalTok{(}\AttributeTok{y =}\NormalTok{ media\_pop, }\AttributeTok{x =} \FunctionTok{as.factor}\NormalTok{(vitoria\_pp))) }\SpecialCharTok{+}
  \FunctionTok{geom\_col}\NormalTok{() }\SpecialCharTok{+}
  \FunctionTok{theme\_minimal}\NormalTok{() }\SpecialCharTok{+} \FunctionTok{guides}\NormalTok{(}\AttributeTok{color =} \StringTok{"none"}\NormalTok{) }\SpecialCharTok{+}
  \FunctionTok{labs}\NormalTok{(}\AttributeTok{title =} \StringTok{""}\NormalTok{,}
       \AttributeTok{y =} \StringTok{"População"}\NormalTok{,}
       \AttributeTok{x =} \StringTok{"Prefeito do PP?"}\NormalTok{)}

\NormalTok{g2 }\OtherTok{\textless{}{-}}\NormalTok{ medias\_municipios\_eleicoes }\SpecialCharTok{\%\textgreater{}\%}
  \FunctionTok{ggplot}\NormalTok{(}\FunctionTok{aes}\NormalTok{(}\AttributeTok{y =}\NormalTok{ media\_densidade\_demografica, }\AttributeTok{x =} \FunctionTok{as.factor}\NormalTok{(vitoria\_pp))) }\SpecialCharTok{+}
  \FunctionTok{geom\_col}\NormalTok{() }\SpecialCharTok{+}
  \FunctionTok{theme\_minimal}\NormalTok{() }\SpecialCharTok{+}
  \FunctionTok{labs}\NormalTok{(}\AttributeTok{title =} \StringTok{""}\NormalTok{,}
       \AttributeTok{y =} \StringTok{"Densidade}\SpecialCharTok{\textbackslash{}n}\StringTok{demográfica"}\NormalTok{,}
       \AttributeTok{x =} \StringTok{"Prefeito do PP?"}\NormalTok{)}

\NormalTok{g3 }\OtherTok{\textless{}{-}}\NormalTok{ medias\_municipios\_eleicoes }\SpecialCharTok{\%\textgreater{}\%}
  \FunctionTok{ggplot}\NormalTok{(}\FunctionTok{aes}\NormalTok{(}\AttributeTok{y =}\NormalTok{ media\_PIBpc, }\AttributeTok{x =} \FunctionTok{as.factor}\NormalTok{(vitoria\_pp))) }\SpecialCharTok{+}
  \FunctionTok{geom\_col}\NormalTok{() }\SpecialCharTok{+}
  \FunctionTok{theme\_minimal}\NormalTok{() }\SpecialCharTok{+}
  \FunctionTok{labs}\NormalTok{(}\AttributeTok{title =} \StringTok{""}\NormalTok{,}
       \AttributeTok{y =} \StringTok{"PIB per capita"}\NormalTok{,}
       \AttributeTok{x =} \StringTok{"Prefeito do PP?"}\NormalTok{)}

\NormalTok{g4 }\OtherTok{\textless{}{-}}\NormalTok{ medias\_municipios\_eleicoes }\SpecialCharTok{\%\textgreater{}\%}
  \FunctionTok{ggplot}\NormalTok{(}\FunctionTok{aes}\NormalTok{(}\AttributeTok{y =}\NormalTok{ media\_taxa\_escolarizacao, }\AttributeTok{x =} \FunctionTok{as.factor}\NormalTok{(vitoria\_pp))) }\SpecialCharTok{+}
  \FunctionTok{geom\_col}\NormalTok{() }\SpecialCharTok{+}
  \FunctionTok{theme\_minimal}\NormalTok{() }\SpecialCharTok{+}
  \FunctionTok{labs}\NormalTok{(}\AttributeTok{title =} \StringTok{""}\NormalTok{,}
       \AttributeTok{y =} \StringTok{"Taxa de escolarização"}\NormalTok{,}
       \AttributeTok{x =} \StringTok{"Prefeito do PP?"}\NormalTok{)}

\NormalTok{g5 }\OtherTok{\textless{}{-}}\NormalTok{ medias\_municipios\_eleicoes }\SpecialCharTok{\%\textgreater{}\%}
  \FunctionTok{ggplot}\NormalTok{(}\FunctionTok{aes}\NormalTok{(}\AttributeTok{y =}\NormalTok{ media\_mortalidade\_infantil, }\AttributeTok{x =} \FunctionTok{as.factor}\NormalTok{(vitoria\_pp))) }\SpecialCharTok{+}
  \FunctionTok{geom\_col}\NormalTok{() }\SpecialCharTok{+}
  \FunctionTok{theme\_minimal}\NormalTok{() }\SpecialCharTok{+}
  \FunctionTok{labs}\NormalTok{(}\AttributeTok{title =} \StringTok{""}\NormalTok{,}
       \AttributeTok{y =} \StringTok{"Mortalidade infantil"}\NormalTok{,}
       \AttributeTok{x =} \StringTok{"Prefeito do PP?"}\NormalTok{)}

\NormalTok{graficos }\OtherTok{\textless{}{-}} \FunctionTok{plot\_grid}\NormalTok{(g1, g2, g3, g4, g5)}

\CommentTok{\# plot\_grid(graficos, ncol = 1, rel\_heights = c(0.15, 1))}
\NormalTok{graficos}
\end{Highlighting}
\end{Shaded}

\includegraphics{estimacao_principal_brasil_20251123_files/figure-pdf/grafico_barras_medias_caracteristicas_municipais-1.pdf}

\begin{Shaded}
\begin{Highlighting}[]
\FunctionTok{rm}\NormalTok{(medias\_municipios\_eleicoes, titulo, graficos, g1, g2, g3, g4, g5)}

\CommentTok{\# Mostrar quais diferenças são estatisticamente significantes {-} fazer teste de comparação des}
\end{Highlighting}
\end{Shaded}

\begin{itemize}
\tightlist
\item
  As médias de população e Densidade\textbackslash ndemográfica são
  bastante diferentes entre os dois grupos. Vamos olhar para a mediana,
  que reduz o efeito de outliers:
\end{itemize}

\paragraph{2.2.1.1) Via medianas}\label{via-medianas}

\begin{Shaded}
\begin{Highlighting}[]
\CommentTok{\# Criar dt com medianas}
\NormalTok{medianas\_municipios\_eleicoes }\OtherTok{\textless{}{-}}\NormalTok{ dt\_municipios\_eleicoes[}
\NormalTok{  ,}
\NormalTok{  .(}
    \AttributeTok{mediana\_pop =} \FunctionTok{median}\NormalTok{(pop),}
    \AttributeTok{mediana\_densidade\_demografica =} \FunctionTok{median}\NormalTok{(densidade\_demografica),}
    \AttributeTok{mediana\_PIBpc =} \FunctionTok{median}\NormalTok{(PIBpc),}
    \AttributeTok{mediana\_taxa\_escolarizacao =} \FunctionTok{median}\NormalTok{(taxa\_escolarizacao),}
    \AttributeTok{mediana\_mortalidade\_infantil =} \FunctionTok{median}\NormalTok{(mortalidade\_infantil, }\AttributeTok{na.rm =}\NormalTok{ T)}
\NormalTok{  ),}
\NormalTok{  by }\OtherTok{=}\NormalTok{ vitoria\_pp}
\NormalTok{]}

\CommentTok{\# Plotar medianas de quem venceu vs. não venceu {-} formatar}
\NormalTok{titulo }\OtherTok{\textless{}{-}} \FunctionTok{ggplot}\NormalTok{() }\SpecialCharTok{+} 
  \FunctionTok{labs}\NormalTok{(}\AttributeTok{title =} \StringTok{"Comparação das medianas de características municipais"}\NormalTok{, }\AttributeTok{subtitle =} \StringTok{"Fontes dos dados: TSE (2021), IBGE (2021, 2022). Elaboração própria."}\NormalTok{) }\SpecialCharTok{+} 
  \FunctionTok{theme\_minimal}\NormalTok{()}

\NormalTok{g1 }\OtherTok{\textless{}{-}}\NormalTok{ medianas\_municipios\_eleicoes }\SpecialCharTok{\%\textgreater{}\%}
  \FunctionTok{ggplot}\NormalTok{(}\FunctionTok{aes}\NormalTok{(}\AttributeTok{y =}\NormalTok{ mediana\_pop, }\AttributeTok{x =} \FunctionTok{as.factor}\NormalTok{(vitoria\_pp), }\AttributeTok{data =}\NormalTok{ )) }\SpecialCharTok{+}
  \FunctionTok{geom\_col}\NormalTok{() }\SpecialCharTok{+}
  \FunctionTok{theme\_minimal}\NormalTok{() }\SpecialCharTok{+}
  \FunctionTok{labs}\NormalTok{(}\AttributeTok{title =} \StringTok{""}\NormalTok{,}
       \AttributeTok{y =} \StringTok{"População"}\NormalTok{,}
       \AttributeTok{x =} \StringTok{"Prefeito do PP?"}\NormalTok{)}

\NormalTok{g2 }\OtherTok{\textless{}{-}}\NormalTok{ medianas\_municipios\_eleicoes }\SpecialCharTok{\%\textgreater{}\%}
  \FunctionTok{ggplot}\NormalTok{(}\FunctionTok{aes}\NormalTok{(}\AttributeTok{y =}\NormalTok{ mediana\_densidade\_demografica, }\AttributeTok{x =} \FunctionTok{as.factor}\NormalTok{(vitoria\_pp))) }\SpecialCharTok{+}
  \FunctionTok{geom\_col}\NormalTok{() }\SpecialCharTok{+}
  \FunctionTok{theme\_minimal}\NormalTok{() }\SpecialCharTok{+}
  \FunctionTok{labs}\NormalTok{(}\AttributeTok{title =} \StringTok{""}\NormalTok{,}
       \AttributeTok{y =} \StringTok{"Densidade}\SpecialCharTok{\textbackslash{}n}\StringTok{demográfica"}\NormalTok{,}
       \AttributeTok{x =} \StringTok{"Prefeito do PP?"}\NormalTok{)}

\NormalTok{g3 }\OtherTok{\textless{}{-}}\NormalTok{ medianas\_municipios\_eleicoes }\SpecialCharTok{\%\textgreater{}\%}
  \FunctionTok{ggplot}\NormalTok{(}\FunctionTok{aes}\NormalTok{(}\AttributeTok{y =}\NormalTok{ mediana\_PIBpc, }\AttributeTok{x =} \FunctionTok{as.factor}\NormalTok{(vitoria\_pp))) }\SpecialCharTok{+}
  \FunctionTok{geom\_col}\NormalTok{() }\SpecialCharTok{+}
  \FunctionTok{theme\_minimal}\NormalTok{() }\SpecialCharTok{+}
  \FunctionTok{labs}\NormalTok{(}\AttributeTok{title =} \StringTok{""}\NormalTok{,}
       \AttributeTok{y =} \StringTok{"PIB per capita"}\NormalTok{,}
       \AttributeTok{x =} \StringTok{"Prefeito do PP?"}\NormalTok{)}

\NormalTok{g4 }\OtherTok{\textless{}{-}}\NormalTok{ medianas\_municipios\_eleicoes }\SpecialCharTok{\%\textgreater{}\%}
  \FunctionTok{ggplot}\NormalTok{(}\FunctionTok{aes}\NormalTok{(}\AttributeTok{y =}\NormalTok{ mediana\_taxa\_escolarizacao, }\AttributeTok{x =} \FunctionTok{as.factor}\NormalTok{(vitoria\_pp))) }\SpecialCharTok{+}
  \FunctionTok{geom\_col}\NormalTok{() }\SpecialCharTok{+}
  \FunctionTok{theme\_minimal}\NormalTok{() }\SpecialCharTok{+}
  \FunctionTok{labs}\NormalTok{(}\AttributeTok{title =} \StringTok{""}\NormalTok{,}
       \AttributeTok{y =} \StringTok{"Taxa de escolarização"}\NormalTok{,}
       \AttributeTok{x =} \StringTok{"Prefeito do PP?"}\NormalTok{)}

\NormalTok{g5 }\OtherTok{\textless{}{-}}\NormalTok{ medianas\_municipios\_eleicoes }\SpecialCharTok{\%\textgreater{}\%}
  \FunctionTok{ggplot}\NormalTok{(}\FunctionTok{aes}\NormalTok{(}\AttributeTok{y =}\NormalTok{ mediana\_mortalidade\_infantil, }\AttributeTok{x =} \FunctionTok{as.factor}\NormalTok{(vitoria\_pp))) }\SpecialCharTok{+}
  \FunctionTok{geom\_col}\NormalTok{() }\SpecialCharTok{+}
  \FunctionTok{theme\_minimal}\NormalTok{() }\SpecialCharTok{+}
  \FunctionTok{labs}\NormalTok{(}\AttributeTok{title =} \StringTok{""}\NormalTok{,}
       \AttributeTok{y =} \StringTok{"Mortalidade infantil"}\NormalTok{,}
       \AttributeTok{x =} \StringTok{"Prefeito do PP?"}\NormalTok{)}

\NormalTok{graficos }\OtherTok{\textless{}{-}} \FunctionTok{plot\_grid}\NormalTok{(g1, g2, g3, g4, g5)}

\CommentTok{\# plot\_grid(titulo, graficos, ncol = 1, rel\_heights = c(0.15, 1))}
\NormalTok{graficos}
\end{Highlighting}
\end{Shaded}

\includegraphics{estimacao_principal_brasil_20251123_files/figure-pdf/grafico_barras_medianas_caracteristicas_municipais-1.pdf}

\begin{Shaded}
\begin{Highlighting}[]
\FunctionTok{rm}\NormalTok{(medianas\_municipios\_eleicoes, titulo, graficos, g1, g2, g3, g4, g5)}
\end{Highlighting}
\end{Shaded}

\begin{itemize}
\tightlist
\item
  Usando a mediana, as médias de população e
  Densidade\textbackslash ndemográfica ficaram bem mais parecidas.
\end{itemize}

\subsubsection{2.2.2) Valores empenhados de emendas
parlamentares}\label{valores-empenhados-de-emendas-parlamentares}

Vamos calcular a diferença de médias dos valores empenhados per capita
dos municípios que tiveram eleições acirradas e elegeram ou não o
candidato do PP, o que seria equivalente a ATT dado homogeneidade dos
grupos de controle e tratamento.

\begin{Shaded}
\begin{Highlighting}[]
\NormalTok{medias\_valor\_empenhado\_ipca\_pc }\OtherTok{\textless{}{-}}\NormalTok{ dt\_individuais[, .(}\AttributeTok{media\_empenho =} \FunctionTok{mean}\NormalTok{(valor\_empenhado\_ipca\_pc)), by }\OtherTok{=}\NormalTok{ vitoria\_pp]}

\NormalTok{medias\_valor\_empenhado\_ipca\_pc[vitoria\_pp }\SpecialCharTok{==} \DecValTok{1}\NormalTok{, media\_empenho] }\SpecialCharTok{{-}}\NormalTok{ medias\_valor\_empenhado\_ipca\_pc[vitoria\_pp }\SpecialCharTok{==} \DecValTok{0}\NormalTok{, media\_empenho]}
\end{Highlighting}
\end{Shaded}

\begin{verbatim}
[1] -1.54903
\end{verbatim}

\begin{itemize}
\tightlist
\item
  Perceba que é uma diferença pequena e negativa, o que não corrobora
  nossa hipótese.
\end{itemize}

\section{3) Estimação}\label{estimauxe7uxe3o}

Vamos estimar o valor adicional recebido via emendas parlamentares
individuaispelos municípios em que o candidato a prefeito do PP ganhou
por pouco, usando alguns modelos diferentes.

\begin{itemize}
\tightlist
\item
  Note que, quando o modelo inclui a running variable, é sua versão
  centralizada (votos\_razao\_pp\_centr), porque isso evita que o
  intercepto tenha valores sem sentido (como valores empenhados
  negativos).
\end{itemize}

\begin{Shaded}
\begin{Highlighting}[]
\CommentTok{\# Duas regressões lineares com controles de município}
\NormalTok{regs\_lineares\_controles }\OtherTok{\textless{}{-}} \FunctionTok{lm\_robust}\NormalTok{(valor\_empenhado\_ipca\_pc }\SpecialCharTok{\textasciitilde{}}\NormalTok{ vitoria\_pp}\SpecialCharTok{*}\NormalTok{votos\_razao\_pp\_centr }\SpecialCharTok{+}\NormalTok{ regiao }\SpecialCharTok{+}\NormalTok{ pop }\SpecialCharTok{+}\NormalTok{ densidade\_demografica }\SpecialCharTok{+}\NormalTok{ PIBpc }\SpecialCharTok{+}\NormalTok{ taxa\_escolarizacao, }\AttributeTok{data =}\NormalTok{ dt\_individuais)}
\end{Highlighting}
\end{Shaded}

\section{4) Teste de heterogeneidade}\label{teste-de-heterogeneidade}

Seria bom repetir a estimação por região, mas seria preciso prestar
atenção no número de observações e de tratados.

\subsection{4.1) Norte}\label{norte}

\begin{Shaded}
\begin{Highlighting}[]
\CommentTok{\# Filtrar base de dados pra região Norte}

\CommentTok{\# Repetir estimação}
\end{Highlighting}
\end{Shaded}

\subsection{4.2) Nordeste}\label{nordeste}

\begin{Shaded}
\begin{Highlighting}[]
\CommentTok{\# Filtrar base de dados pra região Nordeste}

\CommentTok{\# Repetir estimação}
\end{Highlighting}
\end{Shaded}

\subsection{4.3) Sudeste}\label{sudeste}

\begin{Shaded}
\begin{Highlighting}[]
\CommentTok{\# Filtrar base de dados pra região Sudeste}

\CommentTok{\# Repetir estimação}
\end{Highlighting}
\end{Shaded}

\subsection{4.4) Sul}\label{sul}

\begin{Shaded}
\begin{Highlighting}[]
\CommentTok{\# Filtrar base de dados pra região Sul}

\CommentTok{\# Repetir estimação}
\end{Highlighting}
\end{Shaded}

\subsection{4.5) Centro-Oeste}\label{centro-oeste}

\begin{Shaded}
\begin{Highlighting}[]
\CommentTok{\# Filtrar base de dados pra região Centro{-}Oeste}

\CommentTok{\# Repetir estimação}
\end{Highlighting}
\end{Shaded}

\section{5) Teste placebo}\label{teste-placebo}

Não vamos fazer, já que não encontramos efeito. (Se encontramos efeito
para certa região, fazer para ela.)

\section{6) Teste de robustez: Modelo linear sem
controles}\label{teste-de-robustez-modelo-linear-sem-controles}

Vamos repetir as estatísticas e estimação para o modelo linear sem
controles.

\subsection{6.1) Verificação de
(des)continuidades}\label{verificauxe7uxe3o-de-descontinuidades-1}

\begin{Shaded}
\begin{Highlighting}[]
\CommentTok{\# Recuperação das versões completas das bases de dados}
\NormalTok{dt\_todas }\OtherTok{\textless{}{-}}\NormalTok{ dt\_todas\_completa}
\NormalTok{dt\_individuais }\OtherTok{\textless{}{-}}\NormalTok{ dt\_individuais\_completa}
\NormalTok{dt\_bancada }\OtherTok{\textless{}{-}}\NormalTok{ dt\_bancada\_completa}
\NormalTok{dt\_relator }\OtherTok{\textless{}{-}}\NormalTok{ dt\_relator\_completa}
\end{Highlighting}
\end{Shaded}

\subsection{6.2) Comparação entre tratados e
controles}\label{comparauxe7uxe3o-entre-tratados-e-controles-1}

\subsubsection{6.2.0) Obter janela ótima e filtrar bases de
dados}\label{obter-janela-uxf3tima-e-filtrar-bases-de-dados-1}

Agora, vamos mostrar que os municípios em que o candidato do PP ganhou
vs.~perdeu por pouco de fato são semelhantes. Para isso, vamos filtrar
as bases de dados por eleições acirradas.

\begin{Shaded}
\begin{Highlighting}[]
\CommentTok{\# Obter janela ótima para modelo linear sem covariadas}
\NormalTok{janela }\OtherTok{\textless{}{-}} \FunctionTok{rdbwselect}\NormalTok{(}
  \AttributeTok{x =}\NormalTok{ dt\_individuais}\SpecialCharTok{$}\NormalTok{votos\_razao\_pp\_centr,}
  \AttributeTok{y =}\NormalTok{ dt\_individuais}\SpecialCharTok{$}\NormalTok{valor\_empenhado\_ipca\_pc,}
  \AttributeTok{p =} \DecValTok{1}
\NormalTok{)}
\NormalTok{janela}\SpecialCharTok{$}\NormalTok{bws}
\end{Highlighting}
\end{Shaded}

\begin{verbatim}
      h (left) h (right)  b (left) b (right)
mserd 0.107061  0.107061 0.2173563 0.2173563
\end{verbatim}

\begin{Shaded}
\begin{Highlighting}[]
\CommentTok{\# Filtrar base de dados para municípios cuja margem de vitória do PP está na janela}
\NormalTok{dt\_todas }\OtherTok{\textless{}{-}}\NormalTok{ dt\_todas[votos\_razao\_pp\_centr }\SpecialCharTok{\textgreater{}=} \SpecialCharTok{{-}}\NormalTok{janela}\SpecialCharTok{$}\NormalTok{bws[}\DecValTok{1}\NormalTok{] }\SpecialCharTok{\&}
\NormalTok{                       votos\_razao\_pp\_centr }\SpecialCharTok{\textless{}=}\NormalTok{ janela}\SpecialCharTok{$}\NormalTok{bws[}\DecValTok{1}\NormalTok{]]}
\NormalTok{dt\_individuais }\OtherTok{\textless{}{-}}\NormalTok{ dt\_individuais[votos\_razao\_pp\_centr }\SpecialCharTok{\textgreater{}=} \SpecialCharTok{{-}}\NormalTok{janela}\SpecialCharTok{$}\NormalTok{bws[}\DecValTok{1}\NormalTok{] }\SpecialCharTok{\&}
\NormalTok{                                   votos\_razao\_pp\_centr }\SpecialCharTok{\textless{}=}\NormalTok{ janela}\SpecialCharTok{$}\NormalTok{bws[}\DecValTok{1}\NormalTok{]]}
\NormalTok{dt\_bancada }\OtherTok{\textless{}{-}}\NormalTok{ dt\_bancada[votos\_razao\_pp\_centr }\SpecialCharTok{\textgreater{}=} \SpecialCharTok{{-}}\NormalTok{janela}\SpecialCharTok{$}\NormalTok{bws[}\DecValTok{1}\NormalTok{] }\SpecialCharTok{\&}
\NormalTok{                           votos\_razao\_pp\_centr }\SpecialCharTok{\textless{}=}\NormalTok{ janela}\SpecialCharTok{$}\NormalTok{bws[}\DecValTok{1}\NormalTok{]]}
\NormalTok{dt\_relator }\OtherTok{\textless{}{-}}\NormalTok{ dt\_relator[votos\_razao\_pp\_centr }\SpecialCharTok{\textgreater{}=} \SpecialCharTok{{-}}\NormalTok{janela}\SpecialCharTok{$}\NormalTok{bws[}\DecValTok{1}\NormalTok{] }\SpecialCharTok{\&}
\NormalTok{                           votos\_razao\_pp\_centr }\SpecialCharTok{\textless{}=}\NormalTok{ janela}\SpecialCharTok{$}\NormalTok{bws[}\DecValTok{1}\NormalTok{]]}

\CommentTok{\# Ver quantos municípios com eleições acirradas receberam emendas}
\FunctionTok{writeLines}\NormalTok{(}\FunctionTok{paste0}\NormalTok{(}
  \StringTok{"Quantidade de municípios com eleições acirradas: "}\NormalTok{, }
  \FunctionTok{nrow}\NormalTok{(dt\_todas),}
  
  \StringTok{"}\SpecialCharTok{\textbackslash{}n}\StringTok{Quantidade de municípios com eleições acirradas que receberam emendas: "}\NormalTok{, }
  \FunctionTok{nrow}\NormalTok{(dt\_todas[valor\_empenhado\_ipca }\SpecialCharTok{!=} \DecValTok{0}\NormalTok{]),}

  \StringTok{"}\SpecialCharTok{\textbackslash{}n}\StringTok{Quantidade de municípios com eleições acirradas que receberam emendas individuais: "}\NormalTok{,}
  \FunctionTok{nrow}\NormalTok{(dt\_individuais[valor\_empenhado\_ipca }\SpecialCharTok{!=} \DecValTok{0}\NormalTok{]),}
  
  \StringTok{"}\SpecialCharTok{\textbackslash{}n}\StringTok{Quantidade de municípios com eleições acirradas que receberam emendas de bancada: "}\NormalTok{,}
  \FunctionTok{nrow}\NormalTok{(dt\_bancada[valor\_empenhado\_ipca }\SpecialCharTok{!=} \DecValTok{0}\NormalTok{]),}
  
  \StringTok{"}\SpecialCharTok{\textbackslash{}n}\StringTok{Quantidade de municípios com eleições acirradas que receberam emendas de relator: "}\NormalTok{,}
  \FunctionTok{nrow}\NormalTok{(dt\_relator[valor\_empenhado\_ipca }\SpecialCharTok{!=} \DecValTok{0}\NormalTok{])))}
\end{Highlighting}
\end{Shaded}

\begin{verbatim}
Quantidade de municípios com eleições acirradas: 815
Quantidade de municípios com eleições acirradas que receberam emendas: 70
Quantidade de municípios com eleições acirradas que receberam emendas individuais: 65
Quantidade de municípios com eleições acirradas que receberam emendas de bancada: 9
Quantidade de municípios com eleições acirradas que receberam emendas de relator: 0
\end{verbatim}

\begin{itemize}
\tightlist
\item
  Note que nenhum município com eleições acirradas recebeu emendas de
  relator e somente 5 receberam emendas de bancada, então vamos analisar
  somente emendas individuais.
\end{itemize}

\subsubsection{6.2.1) Características
municipais}\label{caracteruxedsticas-municipais-3}

\paragraph{6.2.1.1) Via médias}\label{via-muxe9dias-1}

\begin{Shaded}
\begin{Highlighting}[]
\CommentTok{\# Criar dt com médias}
\NormalTok{medias\_municipios\_eleicoes }\OtherTok{\textless{}{-}}\NormalTok{ dt\_municipios\_eleicoes[}
\NormalTok{  ,}
\NormalTok{  .(}
    \AttributeTok{media\_pop =} \FunctionTok{mean}\NormalTok{(pop),}
    \AttributeTok{media\_densidade\_demografica =} \FunctionTok{mean}\NormalTok{(densidade\_demografica),}
    \AttributeTok{media\_PIBpc =} \FunctionTok{mean}\NormalTok{(PIBpc),}
    \AttributeTok{media\_taxa\_escolarizacao =} \FunctionTok{mean}\NormalTok{(taxa\_escolarizacao),}
    \AttributeTok{media\_mortalidade\_infantil =} \FunctionTok{mean}\NormalTok{(mortalidade\_infantil, }\AttributeTok{na.rm =}\NormalTok{ T)}
\NormalTok{  ),}
\NormalTok{  by }\OtherTok{=}\NormalTok{ vitoria\_pp}
\NormalTok{]}

\CommentTok{\# Plotar médias de quem venceu vs. não venceu {-} formatar}
\NormalTok{titulo }\OtherTok{\textless{}{-}} \FunctionTok{ggplot}\NormalTok{() }\SpecialCharTok{+} 
  \FunctionTok{labs}\NormalTok{(}\AttributeTok{title =} \StringTok{"Comparação das médias de características municipais"}\NormalTok{, }\AttributeTok{subtitle =} \StringTok{"Fontes dos dados: TSE, IBGE (2021, 2022). Elaboração própria."}\NormalTok{) }\SpecialCharTok{+}
  \FunctionTok{theme\_minimal}\NormalTok{()}

\NormalTok{g1 }\OtherTok{\textless{}{-}}\NormalTok{ medias\_municipios\_eleicoes }\SpecialCharTok{\%\textgreater{}\%}
  \FunctionTok{ggplot}\NormalTok{(}\FunctionTok{aes}\NormalTok{(}\AttributeTok{y =}\NormalTok{ media\_pop, }\AttributeTok{x =} \FunctionTok{as.factor}\NormalTok{(vitoria\_pp))) }\SpecialCharTok{+}
  \FunctionTok{geom\_col}\NormalTok{() }\SpecialCharTok{+}
  \FunctionTok{theme\_minimal}\NormalTok{() }\SpecialCharTok{+} \FunctionTok{guides}\NormalTok{(}\AttributeTok{color =} \StringTok{"none"}\NormalTok{) }\SpecialCharTok{+}
  \FunctionTok{labs}\NormalTok{(}\AttributeTok{title =} \StringTok{""}\NormalTok{,}
       \AttributeTok{y =} \StringTok{"População"}\NormalTok{,}
       \AttributeTok{x =} \StringTok{"Prefeito do PP?"}\NormalTok{)}

\NormalTok{g2 }\OtherTok{\textless{}{-}}\NormalTok{ medias\_municipios\_eleicoes }\SpecialCharTok{\%\textgreater{}\%}
  \FunctionTok{ggplot}\NormalTok{(}\FunctionTok{aes}\NormalTok{(}\AttributeTok{y =}\NormalTok{ media\_densidade\_demografica, }\AttributeTok{x =} \FunctionTok{as.factor}\NormalTok{(vitoria\_pp))) }\SpecialCharTok{+}
  \FunctionTok{geom\_col}\NormalTok{() }\SpecialCharTok{+}
  \FunctionTok{theme\_minimal}\NormalTok{() }\SpecialCharTok{+}
  \FunctionTok{labs}\NormalTok{(}\AttributeTok{title =} \StringTok{""}\NormalTok{,}
       \AttributeTok{y =} \StringTok{"Densidade}\SpecialCharTok{\textbackslash{}n}\StringTok{demográfica"}\NormalTok{,}
       \AttributeTok{x =} \StringTok{"Prefeito do PP?"}\NormalTok{)}

\NormalTok{g3 }\OtherTok{\textless{}{-}}\NormalTok{ medias\_municipios\_eleicoes }\SpecialCharTok{\%\textgreater{}\%}
  \FunctionTok{ggplot}\NormalTok{(}\FunctionTok{aes}\NormalTok{(}\AttributeTok{y =}\NormalTok{ media\_PIBpc, }\AttributeTok{x =} \FunctionTok{as.factor}\NormalTok{(vitoria\_pp))) }\SpecialCharTok{+}
  \FunctionTok{geom\_col}\NormalTok{() }\SpecialCharTok{+}
  \FunctionTok{theme\_minimal}\NormalTok{() }\SpecialCharTok{+}
  \FunctionTok{labs}\NormalTok{(}\AttributeTok{title =} \StringTok{""}\NormalTok{,}
       \AttributeTok{y =} \StringTok{"PIB per capita"}\NormalTok{,}
       \AttributeTok{x =} \StringTok{"Prefeito do PP?"}\NormalTok{)}

\NormalTok{g4 }\OtherTok{\textless{}{-}}\NormalTok{ medias\_municipios\_eleicoes }\SpecialCharTok{\%\textgreater{}\%}
  \FunctionTok{ggplot}\NormalTok{(}\FunctionTok{aes}\NormalTok{(}\AttributeTok{y =}\NormalTok{ media\_taxa\_escolarizacao, }\AttributeTok{x =} \FunctionTok{as.factor}\NormalTok{(vitoria\_pp))) }\SpecialCharTok{+}
  \FunctionTok{geom\_col}\NormalTok{() }\SpecialCharTok{+}
  \FunctionTok{theme\_minimal}\NormalTok{() }\SpecialCharTok{+}
  \FunctionTok{labs}\NormalTok{(}\AttributeTok{title =} \StringTok{""}\NormalTok{,}
       \AttributeTok{y =} \StringTok{"Taxa de escolarização"}\NormalTok{,}
       \AttributeTok{x =} \StringTok{"Prefeito do PP?"}\NormalTok{)}

\NormalTok{g5 }\OtherTok{\textless{}{-}}\NormalTok{ medias\_municipios\_eleicoes }\SpecialCharTok{\%\textgreater{}\%}
  \FunctionTok{ggplot}\NormalTok{(}\FunctionTok{aes}\NormalTok{(}\AttributeTok{y =}\NormalTok{ media\_mortalidade\_infantil, }\AttributeTok{x =} \FunctionTok{as.factor}\NormalTok{(vitoria\_pp))) }\SpecialCharTok{+}
  \FunctionTok{geom\_col}\NormalTok{() }\SpecialCharTok{+}
  \FunctionTok{theme\_minimal}\NormalTok{() }\SpecialCharTok{+}
  \FunctionTok{labs}\NormalTok{(}\AttributeTok{title =} \StringTok{""}\NormalTok{,}
       \AttributeTok{y =} \StringTok{"Mortalidade infantil"}\NormalTok{,}
       \AttributeTok{x =} \StringTok{"Prefeito do PP?"}\NormalTok{)}

\NormalTok{graficos }\OtherTok{\textless{}{-}} \FunctionTok{plot\_grid}\NormalTok{(g1, g2, g3, g4, g5)}

\CommentTok{\# plot\_grid(graficos, ncol = 1, rel\_heights = c(0.15, 1))}
\NormalTok{graficos}
\end{Highlighting}
\end{Shaded}

\includegraphics{estimacao_principal_brasil_20251123_files/figure-pdf/grafico_barras_medias_caracteristicas_municipais_linear_sem_controles-1.pdf}

\begin{Shaded}
\begin{Highlighting}[]
\FunctionTok{rm}\NormalTok{(medias\_municipios\_eleicoes, titulo, graficos, g1, g2, g3, g4, g5)}

\CommentTok{\# Mostrar quais diferenças são estatisticamente significantes {-} fazer teste de comparação des}
\end{Highlighting}
\end{Shaded}

\begin{itemize}
\tightlist
\item
  As médias de população e densidade demográfica são bastante diferentes
  entre os dois grupos. Vamos olhar para a mediana, que reduz o efeito
  de outliers:
\end{itemize}

\paragraph{6.2.1.1) Via medianas}\label{via-medianas-1}

\begin{Shaded}
\begin{Highlighting}[]
\CommentTok{\# Criar dt com medianas}
\NormalTok{medianas\_municipios\_eleicoes }\OtherTok{\textless{}{-}}\NormalTok{ dt\_municipios\_eleicoes[}
\NormalTok{  ,}
\NormalTok{  .(}
    \AttributeTok{mediana\_pop =} \FunctionTok{median}\NormalTok{(pop),}
    \AttributeTok{mediana\_densidade\_demografica =} \FunctionTok{median}\NormalTok{(densidade\_demografica),}
    \AttributeTok{mediana\_PIBpc =} \FunctionTok{median}\NormalTok{(PIBpc),}
    \AttributeTok{mediana\_taxa\_escolarizacao =} \FunctionTok{median}\NormalTok{(taxa\_escolarizacao),}
    \AttributeTok{mediana\_mortalidade\_infantil =} \FunctionTok{median}\NormalTok{(mortalidade\_infantil, }\AttributeTok{na.rm =}\NormalTok{ T)}
\NormalTok{  ),}
\NormalTok{  by }\OtherTok{=}\NormalTok{ vitoria\_pp}
\NormalTok{]}

\CommentTok{\# Plotar medianas de quem venceu vs. não venceu {-} formatar}
\NormalTok{titulo }\OtherTok{\textless{}{-}} \FunctionTok{ggplot}\NormalTok{() }\SpecialCharTok{+} 
  \FunctionTok{labs}\NormalTok{(}\AttributeTok{title =} \StringTok{"Comparação das medianas de características municipais"}\NormalTok{, }\AttributeTok{subtitle =} \StringTok{"Fontes dos dados: TSE (2021), IBGE (2021, 2022). Elaboração própria."}\NormalTok{) }\SpecialCharTok{+} 
  \FunctionTok{theme\_minimal}\NormalTok{()}

\NormalTok{g1 }\OtherTok{\textless{}{-}}\NormalTok{ medianas\_municipios\_eleicoes }\SpecialCharTok{\%\textgreater{}\%}
  \FunctionTok{ggplot}\NormalTok{(}\FunctionTok{aes}\NormalTok{(}\AttributeTok{y =}\NormalTok{ mediana\_pop, }\AttributeTok{x =} \FunctionTok{as.factor}\NormalTok{(vitoria\_pp), }\AttributeTok{data =}\NormalTok{ )) }\SpecialCharTok{+}
  \FunctionTok{geom\_col}\NormalTok{() }\SpecialCharTok{+}
  \FunctionTok{theme\_minimal}\NormalTok{() }\SpecialCharTok{+}
  \FunctionTok{labs}\NormalTok{(}\AttributeTok{title =} \StringTok{""}\NormalTok{,}
       \AttributeTok{y =} \StringTok{"População"}\NormalTok{,}
       \AttributeTok{x =} \StringTok{"Prefeito do PP?"}\NormalTok{)}

\NormalTok{g2 }\OtherTok{\textless{}{-}}\NormalTok{ medianas\_municipios\_eleicoes }\SpecialCharTok{\%\textgreater{}\%}
  \FunctionTok{ggplot}\NormalTok{(}\FunctionTok{aes}\NormalTok{(}\AttributeTok{y =}\NormalTok{ mediana\_densidade\_demografica, }\AttributeTok{x =} \FunctionTok{as.factor}\NormalTok{(vitoria\_pp))) }\SpecialCharTok{+}
  \FunctionTok{geom\_col}\NormalTok{() }\SpecialCharTok{+}
  \FunctionTok{theme\_minimal}\NormalTok{() }\SpecialCharTok{+}
  \FunctionTok{labs}\NormalTok{(}\AttributeTok{title =} \StringTok{""}\NormalTok{,}
       \AttributeTok{y =} \StringTok{"Densidade}\SpecialCharTok{\textbackslash{}n}\StringTok{demográfica"}\NormalTok{,}
       \AttributeTok{x =} \StringTok{"Prefeito do PP?"}\NormalTok{)}

\NormalTok{g3 }\OtherTok{\textless{}{-}}\NormalTok{ medianas\_municipios\_eleicoes }\SpecialCharTok{\%\textgreater{}\%}
  \FunctionTok{ggplot}\NormalTok{(}\FunctionTok{aes}\NormalTok{(}\AttributeTok{y =}\NormalTok{ mediana\_PIBpc, }\AttributeTok{x =} \FunctionTok{as.factor}\NormalTok{(vitoria\_pp))) }\SpecialCharTok{+}
  \FunctionTok{geom\_col}\NormalTok{() }\SpecialCharTok{+}
  \FunctionTok{theme\_minimal}\NormalTok{() }\SpecialCharTok{+}
  \FunctionTok{labs}\NormalTok{(}\AttributeTok{title =} \StringTok{""}\NormalTok{,}
       \AttributeTok{y =} \StringTok{"PIB per capita"}\NormalTok{,}
       \AttributeTok{x =} \StringTok{"Prefeito do PP?"}\NormalTok{)}

\NormalTok{g4 }\OtherTok{\textless{}{-}}\NormalTok{ medianas\_municipios\_eleicoes }\SpecialCharTok{\%\textgreater{}\%}
  \FunctionTok{ggplot}\NormalTok{(}\FunctionTok{aes}\NormalTok{(}\AttributeTok{y =}\NormalTok{ mediana\_taxa\_escolarizacao, }\AttributeTok{x =} \FunctionTok{as.factor}\NormalTok{(vitoria\_pp))) }\SpecialCharTok{+}
  \FunctionTok{geom\_col}\NormalTok{() }\SpecialCharTok{+}
  \FunctionTok{theme\_minimal}\NormalTok{() }\SpecialCharTok{+}
  \FunctionTok{labs}\NormalTok{(}\AttributeTok{title =} \StringTok{""}\NormalTok{,}
       \AttributeTok{y =} \StringTok{"Taxa de escolarização"}\NormalTok{,}
       \AttributeTok{x =} \StringTok{"Prefeito do PP?"}\NormalTok{)}

\NormalTok{g5 }\OtherTok{\textless{}{-}}\NormalTok{ medianas\_municipios\_eleicoes }\SpecialCharTok{\%\textgreater{}\%}
  \FunctionTok{ggplot}\NormalTok{(}\FunctionTok{aes}\NormalTok{(}\AttributeTok{y =}\NormalTok{ mediana\_mortalidade\_infantil, }\AttributeTok{x =} \FunctionTok{as.factor}\NormalTok{(vitoria\_pp))) }\SpecialCharTok{+}
  \FunctionTok{geom\_col}\NormalTok{() }\SpecialCharTok{+}
  \FunctionTok{theme\_minimal}\NormalTok{() }\SpecialCharTok{+}
  \FunctionTok{labs}\NormalTok{(}\AttributeTok{title =} \StringTok{""}\NormalTok{,}
       \AttributeTok{y =} \StringTok{"Mortalidade infantil"}\NormalTok{,}
       \AttributeTok{x =} \StringTok{"Prefeito do PP?"}\NormalTok{)}

\NormalTok{graficos }\OtherTok{\textless{}{-}} \FunctionTok{plot\_grid}\NormalTok{(g1, g2, g3, g4, g5)}

\CommentTok{\# plot\_grid(titulo, graficos, ncol = 1, rel\_heights = c(0.15, 1))}
\NormalTok{graficos}
\end{Highlighting}
\end{Shaded}

\includegraphics{estimacao_principal_brasil_20251123_files/figure-pdf/grafico_barras_medianas_caracteristicas_municipais_linear_sem_controles-1.pdf}

\begin{Shaded}
\begin{Highlighting}[]
\FunctionTok{rm}\NormalTok{(medianas\_municipios\_eleicoes, titulo, graficos, g1, g2, g3, g4, g5)}
\end{Highlighting}
\end{Shaded}

\begin{itemize}
\tightlist
\item
  Usando a mediana, as médias de população e densidade demográfica
  ficaram bem mais parecidas.
\end{itemize}

\subsubsection{6.2.2) Valores empenhados de emendas
parlamentares}\label{valores-empenhados-de-emendas-parlamentares-1}

Vamos calcular a diferença de médias dos valores empenhados per capita
dos municípios que tiveram eleições acirradas e elegeram ou não o
candidato do PP, o que seria equivalente a ATT dado homogeneidade dos
grupos de controle e tratamento.

\begin{Shaded}
\begin{Highlighting}[]
\NormalTok{medias\_valor\_empenhado\_ipca\_pc }\OtherTok{\textless{}{-}}\NormalTok{ dt\_individuais[, .(}\AttributeTok{media\_empenho =} \FunctionTok{mean}\NormalTok{(valor\_empenhado\_ipca\_pc)), by }\OtherTok{=}\NormalTok{ vitoria\_pp]}

\NormalTok{medias\_valor\_empenhado\_ipca\_pc[vitoria\_pp }\SpecialCharTok{==} \DecValTok{1}\NormalTok{, media\_empenho] }\SpecialCharTok{{-}}\NormalTok{ medias\_valor\_empenhado\_ipca\_pc[vitoria\_pp }\SpecialCharTok{==} \DecValTok{0}\NormalTok{, media\_empenho]}
\end{Highlighting}
\end{Shaded}

\begin{verbatim}
[1] -1.177472
\end{verbatim}

\begin{itemize}
\tightlist
\item
  Perceba que é uma diferença pequena e negativa, o que não corrobora
  nossa hipótese.
\end{itemize}

\subsection{6.3) Estimação}\label{estimauxe7uxe3o-1}

Vamos estimar o valor adicional recebido via emendas parlamentares
individuaispelos municípios em que o candidato a prefeito do PP ganhou
por pouco, usando alguns modelos diferentes.

\begin{itemize}
\tightlist
\item
  Note que, quando o modelo inclui a running variable, é sua versão
  centralizada (votos\_razao\_pp\_centr), porque isso evita que o
  intercepto tenha valores sem sentido (como valores empenhados
  negativos).
\end{itemize}

\begin{Shaded}
\begin{Highlighting}[]
\CommentTok{\# Duas regressões lineares sem controles de município}
\NormalTok{regs\_lineares }\OtherTok{\textless{}{-}} \FunctionTok{lm\_robust}\NormalTok{(valor\_empenhado\_ipca\_pc }\SpecialCharTok{\textasciitilde{}}\NormalTok{ vitoria\_pp}\SpecialCharTok{*}\NormalTok{votos\_razao\_pp\_centr, }\AttributeTok{data =}\NormalTok{ dt\_individuais)}
\end{Highlighting}
\end{Shaded}

\section{7) Teste de robustez: Modelo
quadrático}\label{teste-de-robustez-modelo-quadruxe1tico}

Vamos repetir as estatísticas e estimação para modelos quadráticos.

\subsection{7.1) Verificação de
(des)continuidades}\label{verificauxe7uxe3o-de-descontinuidades-2}

Note que filtramos os municípios para os percentis 2 a 99 de cada
característica para evitar distorções causadas por outliers.

\subsubsection{7.1.1) Características
municipais}\label{caracteruxedsticas-municipais-4}

\begin{Shaded}
\begin{Highlighting}[]
\CommentTok{\# Recuperação das versões completas das bases de dados}
\NormalTok{dt\_todas }\OtherTok{\textless{}{-}}\NormalTok{ dt\_todas\_completa}
\NormalTok{dt\_individuais }\OtherTok{\textless{}{-}}\NormalTok{ dt\_individuais\_completa}
\NormalTok{dt\_bancada }\OtherTok{\textless{}{-}}\NormalTok{ dt\_bancada\_completa}
\NormalTok{dt\_relator }\OtherTok{\textless{}{-}}\NormalTok{ dt\_relator\_completa}


\CommentTok{\# Linear}
\NormalTok{g\_pop }\OtherTok{\textless{}{-}}\NormalTok{ dt\_todas[pop }\SpecialCharTok{\%between\%} \FunctionTok{quantile}\NormalTok{(pop, }\FunctionTok{c}\NormalTok{(.}\DecValTok{02}\NormalTok{, .}\DecValTok{99}\NormalTok{), }\AttributeTok{na.rm =} \ConstantTok{TRUE}\NormalTok{)] }\SpecialCharTok{\%\textgreater{}\%}
  \FunctionTok{ggplot}\NormalTok{(}\FunctionTok{aes}\NormalTok{(}\AttributeTok{x =}\NormalTok{ votos\_razao\_pp\_centr, }\AttributeTok{y =}\NormalTok{ pop, }\AttributeTok{color =} \FunctionTok{as.factor}\NormalTok{(vitoria\_pp))) }\SpecialCharTok{+}
  \FunctionTok{geom\_point}\NormalTok{(}\AttributeTok{shape =} \DecValTok{21}\NormalTok{, }\AttributeTok{size =} \FloatTok{0.5}\NormalTok{, }\AttributeTok{color =} \StringTok{"black"}\NormalTok{) }\SpecialCharTok{+}
  \FunctionTok{stat\_smooth}\NormalTok{(}\AttributeTok{method =} \StringTok{"lm"}\NormalTok{, }\AttributeTok{se =}\NormalTok{ F, }\AttributeTok{formula =}\NormalTok{ y }\SpecialCharTok{\textasciitilde{}} \FunctionTok{poly}\NormalTok{(x, }\DecValTok{2}\NormalTok{)) }\SpecialCharTok{+}
  \FunctionTok{scale\_color\_brewer}\NormalTok{(}\AttributeTok{palette =} \StringTok{"Dark2"}\NormalTok{, }\AttributeTok{direction =} \SpecialCharTok{{-}}\DecValTok{1}\NormalTok{) }\SpecialCharTok{+}
  \FunctionTok{geom\_vline}\NormalTok{(}\AttributeTok{xintercept =} \FloatTok{0.5}\NormalTok{, }\AttributeTok{linetype =} \StringTok{"dashed"}\NormalTok{) }\SpecialCharTok{+}
  \FunctionTok{scale\_y\_continuous}\NormalTok{(}\AttributeTok{labels =} \FunctionTok{label\_number}\NormalTok{()) }\SpecialCharTok{+}
  \FunctionTok{theme\_minimal}\NormalTok{() }\SpecialCharTok{+} \FunctionTok{guides}\NormalTok{(}\AttributeTok{color =} \StringTok{"none"}\NormalTok{) }\SpecialCharTok{+}
  \FunctionTok{labs}\NormalTok{(}\AttributeTok{title =} \StringTok{""}\NormalTok{,}
       \AttributeTok{y =} \StringTok{"População"}\NormalTok{,}
       \AttributeTok{x =} \StringTok{"Margem de vitória do PP"}\NormalTok{) }\SpecialCharTok{+} \FunctionTok{scale\_x\_continuous}\NormalTok{(}\AttributeTok{labels =} \FunctionTok{label\_percent}\NormalTok{()) }\SpecialCharTok{+} \FunctionTok{scale\_x\_continuous}\NormalTok{(}\AttributeTok{labels =} \FunctionTok{label\_percent}\NormalTok{())}

\NormalTok{g\_densidade\_demografica }\OtherTok{\textless{}{-}}\NormalTok{ dt\_todas[densidade\_demografica }\SpecialCharTok{\%between\%} \FunctionTok{quantile}\NormalTok{(densidade\_demografica, }\FunctionTok{c}\NormalTok{(.}\DecValTok{02}\NormalTok{, .}\DecValTok{99}\NormalTok{), }\AttributeTok{na.rm =} \ConstantTok{TRUE}\NormalTok{)] }\SpecialCharTok{\%\textgreater{}\%}
  \FunctionTok{ggplot}\NormalTok{(}\FunctionTok{aes}\NormalTok{(}\AttributeTok{x =}\NormalTok{ votos\_razao\_pp\_centr, }\AttributeTok{y =}\NormalTok{ densidade\_demografica, }\AttributeTok{color =} \FunctionTok{as.factor}\NormalTok{(vitoria\_pp))) }\SpecialCharTok{+}
  \FunctionTok{geom\_point}\NormalTok{(}\AttributeTok{shape =} \DecValTok{21}\NormalTok{, }\AttributeTok{size =} \FloatTok{0.5}\NormalTok{, }\AttributeTok{color =} \StringTok{"black"}\NormalTok{) }\SpecialCharTok{+}
  \FunctionTok{stat\_smooth}\NormalTok{(}\AttributeTok{method =} \StringTok{"lm"}\NormalTok{, }\AttributeTok{se =}\NormalTok{ F, }\AttributeTok{formula =}\NormalTok{ y }\SpecialCharTok{\textasciitilde{}} \FunctionTok{poly}\NormalTok{(x, }\DecValTok{2}\NormalTok{)) }\SpecialCharTok{+}
  \FunctionTok{scale\_color\_brewer}\NormalTok{(}\AttributeTok{palette =} \StringTok{"Dark2"}\NormalTok{, }\AttributeTok{direction =} \SpecialCharTok{{-}}\DecValTok{1}\NormalTok{) }\SpecialCharTok{+}
  \FunctionTok{geom\_vline}\NormalTok{(}\AttributeTok{xintercept =} \FloatTok{0.5}\NormalTok{, }\AttributeTok{linetype =} \StringTok{"dashed"}\NormalTok{) }\SpecialCharTok{+}
  \FunctionTok{scale\_y\_continuous}\NormalTok{(}\AttributeTok{labels =} \FunctionTok{label\_number}\NormalTok{()) }\SpecialCharTok{+}
  \FunctionTok{theme\_minimal}\NormalTok{() }\SpecialCharTok{+} \FunctionTok{guides}\NormalTok{(}\AttributeTok{color =} \StringTok{"none"}\NormalTok{) }\SpecialCharTok{+}
  \FunctionTok{labs}\NormalTok{(}\AttributeTok{title =} \StringTok{""}\NormalTok{,}
       \AttributeTok{y =} \StringTok{"Densidade}\SpecialCharTok{\textbackslash{}n}\StringTok{demográfica"}\NormalTok{,}
       \AttributeTok{x =} \StringTok{"Margem de vitória do PP"}\NormalTok{) }\SpecialCharTok{+} \FunctionTok{scale\_x\_continuous}\NormalTok{(}\AttributeTok{labels =} \FunctionTok{label\_percent}\NormalTok{())}

\NormalTok{g\_PIBpc }\OtherTok{\textless{}{-}}\NormalTok{ dt\_todas[PIBpc }\SpecialCharTok{\%between\%} \FunctionTok{quantile}\NormalTok{(PIBpc, }\FunctionTok{c}\NormalTok{(.}\DecValTok{02}\NormalTok{, .}\DecValTok{99}\NormalTok{), }\AttributeTok{na.rm =} \ConstantTok{TRUE}\NormalTok{)] }\SpecialCharTok{\%\textgreater{}\%}
  \FunctionTok{ggplot}\NormalTok{(}\FunctionTok{aes}\NormalTok{(}\AttributeTok{x =}\NormalTok{ votos\_razao\_pp\_centr, }\AttributeTok{y =}\NormalTok{ PIBpc, }\AttributeTok{color =} \FunctionTok{as.factor}\NormalTok{(vitoria\_pp))) }\SpecialCharTok{+}
  \FunctionTok{geom\_point}\NormalTok{(}\AttributeTok{shape =} \DecValTok{21}\NormalTok{, }\AttributeTok{size =} \FloatTok{0.5}\NormalTok{, }\AttributeTok{color =} \StringTok{"black"}\NormalTok{) }\SpecialCharTok{+}
  \FunctionTok{stat\_smooth}\NormalTok{(}\AttributeTok{method =} \StringTok{"lm"}\NormalTok{, }\AttributeTok{se =}\NormalTok{ F, }\AttributeTok{formula =}\NormalTok{ y }\SpecialCharTok{\textasciitilde{}} \FunctionTok{poly}\NormalTok{(x, }\DecValTok{2}\NormalTok{)) }\SpecialCharTok{+}
  \FunctionTok{scale\_color\_brewer}\NormalTok{(}\AttributeTok{palette =} \StringTok{"Dark2"}\NormalTok{, }\AttributeTok{direction =} \SpecialCharTok{{-}}\DecValTok{1}\NormalTok{) }\SpecialCharTok{+}
  \FunctionTok{geom\_vline}\NormalTok{(}\AttributeTok{xintercept =} \FloatTok{0.5}\NormalTok{, }\AttributeTok{linetype =} \StringTok{"dashed"}\NormalTok{) }\SpecialCharTok{+}
  \FunctionTok{scale\_y\_continuous}\NormalTok{(}\AttributeTok{labels =} \FunctionTok{label\_number}\NormalTok{()) }\SpecialCharTok{+}
  \FunctionTok{theme\_minimal}\NormalTok{() }\SpecialCharTok{+} \FunctionTok{guides}\NormalTok{(}\AttributeTok{color =} \StringTok{"none"}\NormalTok{) }\SpecialCharTok{+}
  \FunctionTok{labs}\NormalTok{(}\AttributeTok{title =} \StringTok{""}\NormalTok{,}
       \AttributeTok{y =} \StringTok{"PIB per capita"}\NormalTok{,}
       \AttributeTok{x =} \StringTok{"Margem de vitória do PP"}\NormalTok{) }\SpecialCharTok{+} \FunctionTok{scale\_x\_continuous}\NormalTok{(}\AttributeTok{labels =} \FunctionTok{label\_percent}\NormalTok{())}

\NormalTok{g\_taxa\_escolarizacao }\OtherTok{\textless{}{-}}\NormalTok{ dt\_todas[taxa\_escolarizacao }\SpecialCharTok{\%between\%} \FunctionTok{quantile}\NormalTok{(taxa\_escolarizacao, }\FunctionTok{c}\NormalTok{(.}\DecValTok{02}\NormalTok{, .}\DecValTok{99}\NormalTok{), }\AttributeTok{na.rm =} \ConstantTok{TRUE}\NormalTok{)] }\SpecialCharTok{\%\textgreater{}\%}
  \FunctionTok{ggplot}\NormalTok{(}\FunctionTok{aes}\NormalTok{(}\AttributeTok{x =}\NormalTok{ votos\_razao\_pp\_centr, }\AttributeTok{y =}\NormalTok{ taxa\_escolarizacao, }\AttributeTok{color =} \FunctionTok{as.factor}\NormalTok{(vitoria\_pp))) }\SpecialCharTok{+}
  \FunctionTok{geom\_point}\NormalTok{(}\AttributeTok{shape =} \DecValTok{21}\NormalTok{, }\AttributeTok{size =} \FloatTok{0.5}\NormalTok{, }\AttributeTok{color =} \StringTok{"black"}\NormalTok{) }\SpecialCharTok{+}
  \FunctionTok{stat\_smooth}\NormalTok{(}\AttributeTok{method =} \StringTok{"lm"}\NormalTok{, }\AttributeTok{se =}\NormalTok{ F, }\AttributeTok{formula =}\NormalTok{ y }\SpecialCharTok{\textasciitilde{}} \FunctionTok{poly}\NormalTok{(x, }\DecValTok{2}\NormalTok{)) }\SpecialCharTok{+}
  \FunctionTok{scale\_color\_brewer}\NormalTok{(}\AttributeTok{palette =} \StringTok{"Dark2"}\NormalTok{, }\AttributeTok{direction =} \SpecialCharTok{{-}}\DecValTok{1}\NormalTok{) }\SpecialCharTok{+}
  \FunctionTok{geom\_vline}\NormalTok{(}\AttributeTok{xintercept =} \FloatTok{0.5}\NormalTok{, }\AttributeTok{linetype =} \StringTok{"dashed"}\NormalTok{) }\SpecialCharTok{+}
  \FunctionTok{scale\_y\_continuous}\NormalTok{(}\AttributeTok{labels =} \FunctionTok{label\_number}\NormalTok{()) }\SpecialCharTok{+}
  \FunctionTok{theme\_minimal}\NormalTok{() }\SpecialCharTok{+} \FunctionTok{guides}\NormalTok{(}\AttributeTok{color =} \StringTok{"none"}\NormalTok{) }\SpecialCharTok{+}
  \FunctionTok{labs}\NormalTok{(}\AttributeTok{title =} \StringTok{""}\NormalTok{,}
       \AttributeTok{y =} \StringTok{"Taxa de}\SpecialCharTok{\textbackslash{}n}\StringTok{escolarização"}\NormalTok{,}
       \AttributeTok{x =} \StringTok{"Margem de vitória do PP"}\NormalTok{) }\SpecialCharTok{+} \FunctionTok{scale\_x\_continuous}\NormalTok{(}\AttributeTok{labels =} \FunctionTok{label\_percent}\NormalTok{())}

\FunctionTok{plot\_grid}\NormalTok{(g\_pop, g\_densidade\_demografica, g\_PIBpc, g\_taxa\_escolarizacao, }\AttributeTok{ncol =} \DecValTok{2}\NormalTok{)}
\end{Highlighting}
\end{Shaded}

\includegraphics{estimacao_principal_brasil_20251123_files/figure-pdf/grafico_caracteristicas_municipais_por_razao_votos_pp_quad-1.pdf}

\subsubsection{\texorpdfstring{7.1.2) Valor empenhado \emph{per capita}
de emendas
parlamentares}{7.1.2) Valor empenhado per capita de emendas parlamentares}}\label{valor-empenhado-per-capita-de-emendas-parlamentares-1}

\begin{Shaded}
\begin{Highlighting}[]
\CommentTok{\# Linear}
\NormalTok{dt\_individuais[valor\_empenhado\_ipca\_pc }\SpecialCharTok{\%between\%} \FunctionTok{quantile}\NormalTok{(valor\_empenhado\_ipca\_pc, }\FunctionTok{c}\NormalTok{(.}\DecValTok{02}\NormalTok{, .}\DecValTok{99}\NormalTok{), }\AttributeTok{na.rm =} \ConstantTok{TRUE}\NormalTok{)] }\SpecialCharTok{\%\textgreater{}\%}
  \FunctionTok{ggplot}\NormalTok{(}\FunctionTok{aes}\NormalTok{(}\AttributeTok{x =}\NormalTok{ votos\_razao\_pp\_centr, }\AttributeTok{y =}\NormalTok{ valor\_empenhado\_ipca\_pc, }\AttributeTok{color =} \FunctionTok{as.factor}\NormalTok{(vitoria\_pp))) }\SpecialCharTok{+}
  \FunctionTok{geom\_point}\NormalTok{(}\AttributeTok{shape =} \DecValTok{21}\NormalTok{, }\AttributeTok{color =} \StringTok{"black"}\NormalTok{) }\SpecialCharTok{+}
  \FunctionTok{stat\_smooth}\NormalTok{(}\AttributeTok{method =} \StringTok{"lm"}\NormalTok{, }\AttributeTok{se =}\NormalTok{ F, }\AttributeTok{formula =}\NormalTok{ y }\SpecialCharTok{\textasciitilde{}} \FunctionTok{poly}\NormalTok{(x, }\DecValTok{2}\NormalTok{)) }\SpecialCharTok{+}
  \FunctionTok{scale\_color\_brewer}\NormalTok{(}\AttributeTok{palette =} \StringTok{"Dark2"}\NormalTok{, }\AttributeTok{direction =} \SpecialCharTok{{-}}\DecValTok{1}\NormalTok{) }\SpecialCharTok{+}
  \FunctionTok{geom\_vline}\NormalTok{(}\AttributeTok{xintercept =} \FloatTok{0.5}\NormalTok{, }\AttributeTok{linetype =} \StringTok{"dashed"}\NormalTok{) }\SpecialCharTok{+}
  \FunctionTok{scale\_y\_continuous}\NormalTok{(}\AttributeTok{labels =} \FunctionTok{label\_number}\NormalTok{()) }\SpecialCharTok{+}
  \FunctionTok{theme\_minimal}\NormalTok{() }\SpecialCharTok{+} \FunctionTok{guides}\NormalTok{(}\AttributeTok{color =} \StringTok{"none"}\NormalTok{) }\SpecialCharTok{+}
  \FunctionTok{labs}\NormalTok{(}\AttributeTok{title =} \StringTok{""}\NormalTok{,}
       \AttributeTok{y =} \StringTok{"Valor empenhado per capita"}\NormalTok{,}
       \AttributeTok{x =} \StringTok{"Margem de vitória do PP"}\NormalTok{) }\SpecialCharTok{+} \FunctionTok{scale\_x\_continuous}\NormalTok{(}\AttributeTok{labels =} \FunctionTok{label\_percent}\NormalTok{())}
\end{Highlighting}
\end{Shaded}

\includegraphics{estimacao_principal_brasil_20251123_files/figure-pdf/grafico_valor_empenhado_por_razao_votos_pp_quad-1.pdf}

\subsection{7.2) Comparação entre tratados e
controles}\label{comparauxe7uxe3o-entre-tratados-e-controles-2}

\subsubsection{7.2.0) Obter janela ótima e filtrar bases de
dados}\label{obter-janela-uxf3tima-e-filtrar-bases-de-dados-2}

Agora, vamos mostrar que os municípios em que o candidato do PP ganhou
vs.~perdeu por pouco de fato são semelhantes. Para isso, vamos filtrar
as bases de dados por eleições acirradas.

\begin{Shaded}
\begin{Highlighting}[]
\CommentTok{\# Obter janela ótima para modelo quadrático}
\DocumentationTok{\#\#\# Sem covariadas}
\NormalTok{janela\_outra }\OtherTok{\textless{}{-}} \FunctionTok{rdbwselect}\NormalTok{(}
  \AttributeTok{x =}\NormalTok{ dt\_individuais}\SpecialCharTok{$}\NormalTok{votos\_razao\_pp\_centr,}
  \AttributeTok{y =}\NormalTok{ dt\_individuais}\SpecialCharTok{$}\NormalTok{valor\_empenhado\_ipca\_pc,}
  \AttributeTok{p =} \DecValTok{2}
\NormalTok{)}
\NormalTok{janela\_outra}\SpecialCharTok{$}\NormalTok{bws}
\end{Highlighting}
\end{Shaded}

\begin{verbatim}
      h (left) h (right)  b (left) b (right)
mserd  0.11978   0.11978 0.1789485 0.1789485
\end{verbatim}

\begin{Shaded}
\begin{Highlighting}[]
\DocumentationTok{\#\#\# Com covariadas}
\NormalTok{covariadas }\OtherTok{\textless{}{-}}\NormalTok{ dt\_individuais[, .(pop, densidade\_demografica, PIBpc, taxa\_escolarizacao)]}
\NormalTok{janela }\OtherTok{\textless{}{-}} \FunctionTok{rdbwselect}\NormalTok{(}
  \AttributeTok{x =}\NormalTok{ dt\_individuais}\SpecialCharTok{$}\NormalTok{votos\_razao\_pp\_centr,}
  \AttributeTok{y =}\NormalTok{ dt\_individuais}\SpecialCharTok{$}\NormalTok{valor\_empenhado\_ipca\_pc,}
  \AttributeTok{p =} \DecValTok{2}\NormalTok{,}
  \AttributeTok{covs =}\NormalTok{ covariadas}
\NormalTok{)}
\NormalTok{janela}\SpecialCharTok{$}\NormalTok{bws}
\end{Highlighting}
\end{Shaded}

\begin{verbatim}
       h (left) h (right)  b (left) b (right)
mserd 0.1450647 0.1450647 0.2261473 0.2261473
\end{verbatim}

\begin{Shaded}
\begin{Highlighting}[]
\CommentTok{\# Filtrar base de dados para municípios cuja margem de vitória do PP está na janela}
\NormalTok{dt\_todas }\OtherTok{\textless{}{-}}\NormalTok{ dt\_todas[votos\_razao\_pp\_centr }\SpecialCharTok{\textgreater{}=} \SpecialCharTok{{-}}\NormalTok{janela}\SpecialCharTok{$}\NormalTok{bws[}\DecValTok{1}\NormalTok{] }\SpecialCharTok{\&}
\NormalTok{                       votos\_razao\_pp\_centr }\SpecialCharTok{\textless{}=}\NormalTok{ janela}\SpecialCharTok{$}\NormalTok{bws[}\DecValTok{1}\NormalTok{]]}
\NormalTok{dt\_individuais }\OtherTok{\textless{}{-}}\NormalTok{ dt\_individuais[votos\_razao\_pp\_centr }\SpecialCharTok{\textgreater{}=} \SpecialCharTok{{-}}\NormalTok{janela}\SpecialCharTok{$}\NormalTok{bws[}\DecValTok{1}\NormalTok{] }\SpecialCharTok{\&}
\NormalTok{                                   votos\_razao\_pp\_centr }\SpecialCharTok{\textless{}=}\NormalTok{ janela}\SpecialCharTok{$}\NormalTok{bws[}\DecValTok{1}\NormalTok{]]}
\NormalTok{dt\_bancada }\OtherTok{\textless{}{-}}\NormalTok{ dt\_bancada[votos\_razao\_pp\_centr }\SpecialCharTok{\textgreater{}=} \SpecialCharTok{{-}}\NormalTok{janela}\SpecialCharTok{$}\NormalTok{bws[}\DecValTok{1}\NormalTok{] }\SpecialCharTok{\&}
\NormalTok{                           votos\_razao\_pp\_centr }\SpecialCharTok{\textless{}=}\NormalTok{ janela}\SpecialCharTok{$}\NormalTok{bws[}\DecValTok{1}\NormalTok{]]}
\NormalTok{dt\_relator }\OtherTok{\textless{}{-}}\NormalTok{ dt\_relator[votos\_razao\_pp\_centr }\SpecialCharTok{\textgreater{}=} \SpecialCharTok{{-}}\NormalTok{janela}\SpecialCharTok{$}\NormalTok{bws[}\DecValTok{1}\NormalTok{] }\SpecialCharTok{\&}
\NormalTok{                           votos\_razao\_pp\_centr }\SpecialCharTok{\textless{}=}\NormalTok{ janela}\SpecialCharTok{$}\NormalTok{bws[}\DecValTok{1}\NormalTok{]]}

\CommentTok{\# Ver quantos municípios com eleições acirradas receberam emendas}
\FunctionTok{writeLines}\NormalTok{(}\FunctionTok{paste0}\NormalTok{(}
  \StringTok{"Quantidade de municípios com eleições acirradas: "}\NormalTok{, }
  \FunctionTok{nrow}\NormalTok{(dt\_todas),}
  
  \StringTok{"}\SpecialCharTok{\textbackslash{}n}\StringTok{Quantidade de municípios com eleições acirradas que receberam emendas: "}\NormalTok{, }
  \FunctionTok{nrow}\NormalTok{(dt\_todas[valor\_empenhado\_ipca }\SpecialCharTok{!=} \DecValTok{0}\NormalTok{]),}

  \StringTok{"}\SpecialCharTok{\textbackslash{}n}\StringTok{Quantidade de municípios com eleições acirradas que receberam emendas individuais: "}\NormalTok{,}
  \FunctionTok{nrow}\NormalTok{(dt\_individuais[valor\_empenhado\_ipca }\SpecialCharTok{!=} \DecValTok{0}\NormalTok{]),}
  
  \StringTok{"}\SpecialCharTok{\textbackslash{}n}\StringTok{Quantidade de municípios com eleições acirradas que receberam emendas de bancada: "}\NormalTok{,}
  \FunctionTok{nrow}\NormalTok{(dt\_bancada[valor\_empenhado\_ipca }\SpecialCharTok{!=} \DecValTok{0}\NormalTok{]),}
  
  \StringTok{"}\SpecialCharTok{\textbackslash{}n}\StringTok{Quantidade de municípios com eleições acirradas que receberam emendas de relator: "}\NormalTok{,}
  \FunctionTok{nrow}\NormalTok{(dt\_relator[valor\_empenhado\_ipca }\SpecialCharTok{!=} \DecValTok{0}\NormalTok{])))}
\end{Highlighting}
\end{Shaded}

\begin{verbatim}
Quantidade de municípios com eleições acirradas: 951
Quantidade de municípios com eleições acirradas que receberam emendas: 86
Quantidade de municípios com eleições acirradas que receberam emendas individuais: 80
Quantidade de municípios com eleições acirradas que receberam emendas de bancada: 12
Quantidade de municípios com eleições acirradas que receberam emendas de relator: 0
\end{verbatim}

\begin{itemize}
\tightlist
\item
  Note que nenhum município com eleições acirradas recebeu emendas de
  relator e somente 5 receberam emendas de bancada, então vamos analisar
  somente emendas individuais.
\end{itemize}

\subsubsection{7.2.1) Características
municipais}\label{caracteruxedsticas-municipais-5}

\paragraph{7.2.1.1) Via médias}\label{via-muxe9dias-2}

\begin{Shaded}
\begin{Highlighting}[]
\CommentTok{\# Criar dt com médias}
\NormalTok{medias\_municipios\_eleicoes }\OtherTok{\textless{}{-}}\NormalTok{ dt\_municipios\_eleicoes[}
\NormalTok{  ,}
\NormalTok{  .(}
    \AttributeTok{media\_pop =} \FunctionTok{mean}\NormalTok{(pop),}
    \AttributeTok{media\_densidade\_demografica =} \FunctionTok{mean}\NormalTok{(densidade\_demografica),}
    \AttributeTok{media\_PIBpc =} \FunctionTok{mean}\NormalTok{(PIBpc),}
    \AttributeTok{media\_taxa\_escolarizacao =} \FunctionTok{mean}\NormalTok{(taxa\_escolarizacao),}
    \AttributeTok{media\_mortalidade\_infantil =} \FunctionTok{mean}\NormalTok{(mortalidade\_infantil, }\AttributeTok{na.rm =}\NormalTok{ T)}
\NormalTok{  ),}
\NormalTok{  by }\OtherTok{=}\NormalTok{ vitoria\_pp}
\NormalTok{]}

\CommentTok{\# Plotar médias de quem venceu vs. não venceu {-} formatar}
\NormalTok{titulo }\OtherTok{\textless{}{-}} \FunctionTok{ggplot}\NormalTok{() }\SpecialCharTok{+} 
  \FunctionTok{labs}\NormalTok{(}\AttributeTok{title =} \StringTok{"Comparação das médias de características municipais"}\NormalTok{, }\AttributeTok{subtitle =} \StringTok{"Fontes dos dados: TSE, IBGE (2021, 2022). Elaboração própria."}\NormalTok{) }\SpecialCharTok{+}
  \FunctionTok{theme\_minimal}\NormalTok{()}

\NormalTok{g1 }\OtherTok{\textless{}{-}}\NormalTok{ medias\_municipios\_eleicoes }\SpecialCharTok{\%\textgreater{}\%}
  \FunctionTok{ggplot}\NormalTok{(}\FunctionTok{aes}\NormalTok{(}\AttributeTok{y =}\NormalTok{ media\_pop, }\AttributeTok{x =} \FunctionTok{as.factor}\NormalTok{(vitoria\_pp))) }\SpecialCharTok{+}
  \FunctionTok{geom\_col}\NormalTok{() }\SpecialCharTok{+}
  \FunctionTok{theme\_minimal}\NormalTok{() }\SpecialCharTok{+} \FunctionTok{guides}\NormalTok{(}\AttributeTok{color =} \StringTok{"none"}\NormalTok{) }\SpecialCharTok{+}
  \FunctionTok{labs}\NormalTok{(}\AttributeTok{title =} \StringTok{""}\NormalTok{,}
       \AttributeTok{y =} \StringTok{"População"}\NormalTok{,}
       \AttributeTok{x =} \StringTok{"Prefeito do PP?"}\NormalTok{)}

\NormalTok{g2 }\OtherTok{\textless{}{-}}\NormalTok{ medias\_municipios\_eleicoes }\SpecialCharTok{\%\textgreater{}\%}
  \FunctionTok{ggplot}\NormalTok{(}\FunctionTok{aes}\NormalTok{(}\AttributeTok{y =}\NormalTok{ media\_densidade\_demografica, }\AttributeTok{x =} \FunctionTok{as.factor}\NormalTok{(vitoria\_pp))) }\SpecialCharTok{+}
  \FunctionTok{geom\_col}\NormalTok{() }\SpecialCharTok{+}
  \FunctionTok{theme\_minimal}\NormalTok{() }\SpecialCharTok{+}
  \FunctionTok{labs}\NormalTok{(}\AttributeTok{title =} \StringTok{""}\NormalTok{,}
       \AttributeTok{y =} \StringTok{"Densidade}\SpecialCharTok{\textbackslash{}n}\StringTok{demográfica"}\NormalTok{,}
       \AttributeTok{x =} \StringTok{"Prefeito do PP?"}\NormalTok{)}

\NormalTok{g3 }\OtherTok{\textless{}{-}}\NormalTok{ medias\_municipios\_eleicoes }\SpecialCharTok{\%\textgreater{}\%}
  \FunctionTok{ggplot}\NormalTok{(}\FunctionTok{aes}\NormalTok{(}\AttributeTok{y =}\NormalTok{ media\_PIBpc, }\AttributeTok{x =} \FunctionTok{as.factor}\NormalTok{(vitoria\_pp))) }\SpecialCharTok{+}
  \FunctionTok{geom\_col}\NormalTok{() }\SpecialCharTok{+}
  \FunctionTok{theme\_minimal}\NormalTok{() }\SpecialCharTok{+}
  \FunctionTok{labs}\NormalTok{(}\AttributeTok{title =} \StringTok{""}\NormalTok{,}
       \AttributeTok{y =} \StringTok{"PIB per capita"}\NormalTok{,}
       \AttributeTok{x =} \StringTok{"Prefeito do PP?"}\NormalTok{)}

\NormalTok{g4 }\OtherTok{\textless{}{-}}\NormalTok{ medias\_municipios\_eleicoes }\SpecialCharTok{\%\textgreater{}\%}
  \FunctionTok{ggplot}\NormalTok{(}\FunctionTok{aes}\NormalTok{(}\AttributeTok{y =}\NormalTok{ media\_taxa\_escolarizacao, }\AttributeTok{x =} \FunctionTok{as.factor}\NormalTok{(vitoria\_pp))) }\SpecialCharTok{+}
  \FunctionTok{geom\_col}\NormalTok{() }\SpecialCharTok{+}
  \FunctionTok{theme\_minimal}\NormalTok{() }\SpecialCharTok{+}
  \FunctionTok{labs}\NormalTok{(}\AttributeTok{title =} \StringTok{""}\NormalTok{,}
       \AttributeTok{y =} \StringTok{"Taxa de escolarização"}\NormalTok{,}
       \AttributeTok{x =} \StringTok{"Prefeito do PP?"}\NormalTok{)}

\NormalTok{g5 }\OtherTok{\textless{}{-}}\NormalTok{ medias\_municipios\_eleicoes }\SpecialCharTok{\%\textgreater{}\%}
  \FunctionTok{ggplot}\NormalTok{(}\FunctionTok{aes}\NormalTok{(}\AttributeTok{y =}\NormalTok{ media\_mortalidade\_infantil, }\AttributeTok{x =} \FunctionTok{as.factor}\NormalTok{(vitoria\_pp))) }\SpecialCharTok{+}
  \FunctionTok{geom\_col}\NormalTok{() }\SpecialCharTok{+}
  \FunctionTok{theme\_minimal}\NormalTok{() }\SpecialCharTok{+}
  \FunctionTok{labs}\NormalTok{(}\AttributeTok{title =} \StringTok{""}\NormalTok{,}
       \AttributeTok{y =} \StringTok{"Mortalidade infantil"}\NormalTok{,}
       \AttributeTok{x =} \StringTok{"Prefeito do PP?"}\NormalTok{)}

\NormalTok{graficos }\OtherTok{\textless{}{-}} \FunctionTok{plot\_grid}\NormalTok{(g1, g2, g3, g4, g5)}

\CommentTok{\# plot\_grid(graficos, ncol = 1, rel\_heights = c(0.15, 1))}
\NormalTok{graficos}
\end{Highlighting}
\end{Shaded}

\includegraphics{estimacao_principal_brasil_20251123_files/figure-pdf/grafico_barras_medias_caracteristicas_municipais_quad-1.pdf}

\begin{Shaded}
\begin{Highlighting}[]
\FunctionTok{rm}\NormalTok{(medias\_municipios\_eleicoes, titulo, graficos, g1, g2, g3, g4, g5)}

\CommentTok{\# Mostrar quais diferenças são estatisticamente significantes {-} fazer teste de comparação des}
\end{Highlighting}
\end{Shaded}

\begin{itemize}
\tightlist
\item
  As médias de população e Densidade\textbackslash ndemográfica são
  bastante diferentes entre os dois grupos. Vamos olhar para a mediana,
  que reduz o efeito de outliers:
\end{itemize}

\paragraph{7.2.1.1) Via medianas}\label{via-medianas-2}

\begin{Shaded}
\begin{Highlighting}[]
\CommentTok{\# Criar dt com medianas}
\NormalTok{medianas\_municipios\_eleicoes }\OtherTok{\textless{}{-}}\NormalTok{ dt\_municipios\_eleicoes[}
\NormalTok{  ,}
\NormalTok{  .(}
    \AttributeTok{mediana\_pop =} \FunctionTok{median}\NormalTok{(pop),}
    \AttributeTok{mediana\_densidade\_demografica =} \FunctionTok{median}\NormalTok{(densidade\_demografica),}
    \AttributeTok{mediana\_PIBpc =} \FunctionTok{median}\NormalTok{(PIBpc),}
    \AttributeTok{mediana\_taxa\_escolarizacao =} \FunctionTok{median}\NormalTok{(taxa\_escolarizacao),}
    \AttributeTok{mediana\_mortalidade\_infantil =} \FunctionTok{median}\NormalTok{(mortalidade\_infantil, }\AttributeTok{na.rm =}\NormalTok{ T)}
\NormalTok{  ),}
\NormalTok{  by }\OtherTok{=}\NormalTok{ vitoria\_pp}
\NormalTok{]}

\CommentTok{\# Plotar medianas de quem venceu vs. não venceu {-} formatar}
\NormalTok{titulo }\OtherTok{\textless{}{-}} \FunctionTok{ggplot}\NormalTok{() }\SpecialCharTok{+} 
  \FunctionTok{labs}\NormalTok{(}\AttributeTok{title =} \StringTok{"Comparação das medianas de características municipais"}\NormalTok{, }\AttributeTok{subtitle =} \StringTok{"Fontes dos dados: TSE (2021), IBGE (2021, 2022). Elaboração própria."}\NormalTok{) }\SpecialCharTok{+} 
  \FunctionTok{theme\_minimal}\NormalTok{()}

\NormalTok{g1 }\OtherTok{\textless{}{-}}\NormalTok{ medianas\_municipios\_eleicoes }\SpecialCharTok{\%\textgreater{}\%}
  \FunctionTok{ggplot}\NormalTok{(}\FunctionTok{aes}\NormalTok{(}\AttributeTok{y =}\NormalTok{ mediana\_pop, }\AttributeTok{x =} \FunctionTok{as.factor}\NormalTok{(vitoria\_pp), }\AttributeTok{data =}\NormalTok{ )) }\SpecialCharTok{+}
  \FunctionTok{geom\_col}\NormalTok{() }\SpecialCharTok{+}
  \FunctionTok{theme\_minimal}\NormalTok{() }\SpecialCharTok{+}
  \FunctionTok{labs}\NormalTok{(}\AttributeTok{title =} \StringTok{""}\NormalTok{,}
       \AttributeTok{y =} \StringTok{"População"}\NormalTok{,}
       \AttributeTok{x =} \StringTok{"Prefeito do PP?"}\NormalTok{)}

\NormalTok{g2 }\OtherTok{\textless{}{-}}\NormalTok{ medianas\_municipios\_eleicoes }\SpecialCharTok{\%\textgreater{}\%}
  \FunctionTok{ggplot}\NormalTok{(}\FunctionTok{aes}\NormalTok{(}\AttributeTok{y =}\NormalTok{ mediana\_densidade\_demografica, }\AttributeTok{x =} \FunctionTok{as.factor}\NormalTok{(vitoria\_pp))) }\SpecialCharTok{+}
  \FunctionTok{geom\_col}\NormalTok{() }\SpecialCharTok{+}
  \FunctionTok{theme\_minimal}\NormalTok{() }\SpecialCharTok{+}
  \FunctionTok{labs}\NormalTok{(}\AttributeTok{title =} \StringTok{""}\NormalTok{,}
       \AttributeTok{y =} \StringTok{"Densidade}\SpecialCharTok{\textbackslash{}n}\StringTok{demográfica"}\NormalTok{,}
       \AttributeTok{x =} \StringTok{"Prefeito do PP?"}\NormalTok{)}

\NormalTok{g3 }\OtherTok{\textless{}{-}}\NormalTok{ medianas\_municipios\_eleicoes }\SpecialCharTok{\%\textgreater{}\%}
  \FunctionTok{ggplot}\NormalTok{(}\FunctionTok{aes}\NormalTok{(}\AttributeTok{y =}\NormalTok{ mediana\_PIBpc, }\AttributeTok{x =} \FunctionTok{as.factor}\NormalTok{(vitoria\_pp))) }\SpecialCharTok{+}
  \FunctionTok{geom\_col}\NormalTok{() }\SpecialCharTok{+}
  \FunctionTok{theme\_minimal}\NormalTok{() }\SpecialCharTok{+}
  \FunctionTok{labs}\NormalTok{(}\AttributeTok{title =} \StringTok{""}\NormalTok{,}
       \AttributeTok{y =} \StringTok{"PIB per capita"}\NormalTok{,}
       \AttributeTok{x =} \StringTok{"Prefeito do PP?"}\NormalTok{)}

\NormalTok{g4 }\OtherTok{\textless{}{-}}\NormalTok{ medianas\_municipios\_eleicoes }\SpecialCharTok{\%\textgreater{}\%}
  \FunctionTok{ggplot}\NormalTok{(}\FunctionTok{aes}\NormalTok{(}\AttributeTok{y =}\NormalTok{ mediana\_taxa\_escolarizacao, }\AttributeTok{x =} \FunctionTok{as.factor}\NormalTok{(vitoria\_pp))) }\SpecialCharTok{+}
  \FunctionTok{geom\_col}\NormalTok{() }\SpecialCharTok{+}
  \FunctionTok{theme\_minimal}\NormalTok{() }\SpecialCharTok{+}
  \FunctionTok{labs}\NormalTok{(}\AttributeTok{title =} \StringTok{""}\NormalTok{,}
       \AttributeTok{y =} \StringTok{"Taxa de escolarização"}\NormalTok{,}
       \AttributeTok{x =} \StringTok{"Prefeito do PP?"}\NormalTok{)}

\NormalTok{g5 }\OtherTok{\textless{}{-}}\NormalTok{ medianas\_municipios\_eleicoes }\SpecialCharTok{\%\textgreater{}\%}
  \FunctionTok{ggplot}\NormalTok{(}\FunctionTok{aes}\NormalTok{(}\AttributeTok{y =}\NormalTok{ mediana\_mortalidade\_infantil, }\AttributeTok{x =} \FunctionTok{as.factor}\NormalTok{(vitoria\_pp))) }\SpecialCharTok{+}
  \FunctionTok{geom\_col}\NormalTok{() }\SpecialCharTok{+}
  \FunctionTok{theme\_minimal}\NormalTok{() }\SpecialCharTok{+}
  \FunctionTok{labs}\NormalTok{(}\AttributeTok{title =} \StringTok{""}\NormalTok{,}
       \AttributeTok{y =} \StringTok{"Mortalidade infantil"}\NormalTok{,}
       \AttributeTok{x =} \StringTok{"Prefeito do PP?"}\NormalTok{)}

\NormalTok{graficos }\OtherTok{\textless{}{-}} \FunctionTok{plot\_grid}\NormalTok{(g1, g2, g3, g4, g5)}

\CommentTok{\# plot\_grid(titulo, graficos, ncol = 1, rel\_heights = c(0.15, 1))}
\NormalTok{graficos}
\end{Highlighting}
\end{Shaded}

\includegraphics{estimacao_principal_brasil_20251123_files/figure-pdf/grafico_barras_medianas_caracteristicas_municipais_quad-1.pdf}

\begin{Shaded}
\begin{Highlighting}[]
\FunctionTok{rm}\NormalTok{(medianas\_municipios\_eleicoes, titulo, graficos, g1, g2, g3, g4, g5)}
\end{Highlighting}
\end{Shaded}

\begin{itemize}
\tightlist
\item
  Usando a mediana, as médias de população e
  Densidade\textbackslash ndemográfica ficaram bem mais parecidas.
\end{itemize}

\subsubsection{7.2.2) Valores empenhados de emendas
parlamentares}\label{valores-empenhados-de-emendas-parlamentares-2}

Vamos calcular a diferença de médias dos valores empenhados per capita
dos municípios que tiveram eleições acirradas e elegeram ou não o
candidato do PP, o que seria equivalente a ATT dado homogeneidade dos
grupos de controle e tratamento.

\begin{Shaded}
\begin{Highlighting}[]
\NormalTok{medias\_valor\_empenhado\_ipca\_pc }\OtherTok{\textless{}{-}}\NormalTok{ dt\_individuais[, .(}\AttributeTok{media\_empenho =} \FunctionTok{mean}\NormalTok{(valor\_empenhado\_ipca\_pc)), by }\OtherTok{=}\NormalTok{ vitoria\_pp]}

\NormalTok{medias\_valor\_empenhado\_ipca\_pc[vitoria\_pp }\SpecialCharTok{==} \DecValTok{1}\NormalTok{, media\_empenho] }\SpecialCharTok{{-}}\NormalTok{ medias\_valor\_empenhado\_ipca\_pc[vitoria\_pp }\SpecialCharTok{==} \DecValTok{0}\NormalTok{, media\_empenho]}
\end{Highlighting}
\end{Shaded}

\begin{verbatim}
[1] -0.8542102
\end{verbatim}

\begin{itemize}
\tightlist
\item
  Perceba que é uma diferença pequena e negativa, o que não corrobora
  nossa hipótese.
\end{itemize}

\subsection{7.3) Estimação}\label{estimauxe7uxe3o-2}

Vamos estimar o valor adicional recebido via emendas parlamentares
individuaispelos municípios em que o candidato a prefeito do PP ganhou
por pouco, usando alguns modelos diferentes.

\begin{itemize}
\tightlist
\item
  Note que, quando o modelo inclui a running variable, é sua versão
  centralizada (votos\_razao\_pp\_centr), porque isso evita que o
  intercepto tenha valores sem sentido (como valores empenhados
  negativos).
\end{itemize}

\begin{Shaded}
\begin{Highlighting}[]
\CommentTok{\# Duas regressões quadráticas com controles de município}
\NormalTok{regs\_quad\_controles }\OtherTok{\textless{}{-}} \FunctionTok{lm\_robust}\NormalTok{(valor\_empenhado\_ipca\_pc }\SpecialCharTok{\textasciitilde{}}\NormalTok{ vitoria\_pp}\SpecialCharTok{*}\NormalTok{votos\_razao\_pp\_centr }\SpecialCharTok{+}\NormalTok{ vitoria\_pp}\SpecialCharTok{*}\NormalTok{votos\_razao\_pp\_centr\_sq }\SpecialCharTok{+}\NormalTok{ regiao }\SpecialCharTok{+}\NormalTok{ pop }\SpecialCharTok{+}\NormalTok{ densidade\_demografica }\SpecialCharTok{+}\NormalTok{ PIBpc }\SpecialCharTok{+}\NormalTok{ taxa\_escolarizacao, }\AttributeTok{data =}\NormalTok{ dt\_individuais)}
\end{Highlighting}
\end{Shaded}

\section{*Tabela final de resultados}\label{tabela-final-de-resultados}

\begin{Shaded}
\begin{Highlighting}[]
\CommentTok{\# Resultados {-} Todos}
\NormalTok{nomes\_variaveis }\OtherTok{\textless{}{-}} \FunctionTok{c}\NormalTok{(}
  \StringTok{"votos\_razao\_pp\_centr\_sq"} \OtherTok{=} \StringTok{"Margem de vitória do PP ao quadrado"}\NormalTok{,}
  \StringTok{"votos\_razao\_pp\_centr"} \OtherTok{=} \StringTok{"Margem de vitória do PP"}\NormalTok{,}
  \StringTok{"vitoria\_pp"} \OtherTok{=} \StringTok{"Prefeito do PP"}
\NormalTok{)}
\NormalTok{gm }\OtherTok{\textless{}{-}}\NormalTok{ tibble}\SpecialCharTok{::}\FunctionTok{tribble}\NormalTok{(}
        \SpecialCharTok{\textasciitilde{}}\NormalTok{raw, }\SpecialCharTok{\textasciitilde{}}\NormalTok{clean, }\SpecialCharTok{\textasciitilde{}}\NormalTok{fmt, }\SpecialCharTok{\textasciitilde{}}\NormalTok{omit,}
        \StringTok{"nobs"}\NormalTok{, }\StringTok{"Número de observações"}\NormalTok{, }\DecValTok{0}\NormalTok{, }\ConstantTok{FALSE}
\NormalTok{)}
\NormalTok{linhas\_adicionais }\OtherTok{\textless{}{-}} \FunctionTok{tribble}\NormalTok{(}\SpecialCharTok{\textasciitilde{}}\NormalTok{variavel, }\SpecialCharTok{\textasciitilde{}}\NormalTok{linear\_controles, }\SpecialCharTok{\textasciitilde{}}\NormalTok{linear, }\SpecialCharTok{\textasciitilde{}}\NormalTok{quad\_controles,}
                  \StringTok{\textquotesingle{}Janela da margem de vitória do PP\textquotesingle{}}\NormalTok{, }\StringTok{\textquotesingle{}±8,4\%\textquotesingle{}}\NormalTok{, }\StringTok{\textquotesingle{}±10,7\%\textquotesingle{}}\NormalTok{, }\StringTok{\textquotesingle{}±14,5\%\textquotesingle{}}\NormalTok{,}
                  \StringTok{\textquotesingle{}Forma funcional\textquotesingle{}}\NormalTok{, }\StringTok{"Linear"}\NormalTok{, }\StringTok{"Linear"}\NormalTok{, }\StringTok{"Quadrático"}\NormalTok{,}
                  \StringTok{\textquotesingle{}Controles municipais\textquotesingle{}}\NormalTok{, }\StringTok{"Sim"}\NormalTok{, }\StringTok{"Não"}\NormalTok{, }\StringTok{"Sim"}\NormalTok{)}
\FunctionTok{modelsummary}\NormalTok{(}\FunctionTok{list}\NormalTok{(}
    \StringTok{"Linear com controles"} \OtherTok{=}\NormalTok{ regs\_lineares\_controles,}
    \StringTok{"Linear sem controles"} \OtherTok{=}\NormalTok{ regs\_lineares,}
    \StringTok{"Quadrático com controles"} \OtherTok{=}\NormalTok{ regs\_quad\_controles}
\NormalTok{  ),}
  \AttributeTok{statistic =} \ConstantTok{NULL}\NormalTok{,}
  \AttributeTok{stars =}\NormalTok{ T,}
  \AttributeTok{coef\_omit =} \DecValTok{4}\SpecialCharTok{:}\DecValTok{11}\NormalTok{,}
  \AttributeTok{gof\_map =}\NormalTok{ gm,}
  \AttributeTok{gof\_omit =} \StringTok{"R2|AIC|BIC|RMSE"}\NormalTok{,}
  \AttributeTok{coef\_rename =}\NormalTok{ nomes\_variaveis,}
  \AttributeTok{add\_rows =}\NormalTok{ linhas\_adicionais,}
  \AttributeTok{output =} \StringTok{"Resultados/ValorEmpenhadoPerCapitaIndividuais.png"}
\NormalTok{)}
\end{Highlighting}
\end{Shaded}





\end{document}
